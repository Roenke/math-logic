\begin{enumerate}

\item Опишите 2-сортную сигнатуру и теорию коммутативных колец с единицей и модулей над ними (определение этих 
понятий легко найти в интернете).

Сигнатура: $R=\{ \{N, P\}, \{
+:(N \amalg P) \times (N \amalg P) \rightarrow (N \amalg P), \
*:(N \amalg P) \times (N \amalg P) \rightarrow (N \amalg P) \}, \ 
0:N, \ 
0:P, \ 
1:N, \ 
1:P
\}$
						   
Теория:
\begin{align*}
	a + b &= b + a \\
	(a + b) + c &= a + (b + c) \\
	a + 0 &= a \\
	a + (-a) &= 0 \\
	a * b &= b * a \\
	(a * b) * c &= a * (b * c) \\
	a * 1 = a \\
	a * (b + c) &= a * b + a * c \\
	(a + b) * c &= a * c + b * c
\end{align*}

Модуль над кольцом, сигнатура: $\{ \{R, M\}, \diamond : M\times M \rightarrow M, \cdot:R\times M \rightarrow M \}$

Модуль над кольцом, теория:
\begin{align*}
a \diamond b &= b \diamond a \\
(r_1 * r_2) \cdot m &= r_1 \cdot (r_2 \cdot m) \\
r\cdot (m_1 \diamond m_2) &= r\cdot m_1 \diamond r \cdot m_2 \\
(r_1 + r_2) \cdot m &= r_1 \cdot m \diamond r_2 \cdot m
\end{align*}


\item Рассмотрим сигнатуру $(\{N\}, \{ 0 : N, S : N \to N, + : N \times N \to N \})$.
    Рассмотрим следующую теорию:
\begin{align*}
0 + y & = y \\
S(x) + y & = S(x + y)
\end{align*}
Докажите, что следующие формулы невыводимы в этой теории
\begin{itemize}
\item $(x + y) + z = x + (y + z)$
\item $x + y = y + x$

$\llb N \rrb = \mathbb{N}, \
0 := 0 \in \mathbb{N},\
S := n \mapsto n + 1, \ 
+ := (l, r) \mapsto r$

Несложно убедиться, что аксиомы из заданной теории выполнены на данной интерпретации $\Rightarrow$ она является 
моделью. Но теорема не верна. Например, $x = 1, y = 2$, $1 + 2 = 2$, $2 + 1 = 1$. Но $1 \neq 2$. Значит теорема в 
данной модели не верна. Значит и формула в исходной теории невыводима.

\end{itemize}
Напомню, что для доказательства невыводимости формулы достаточно привести пример модели в которой эта формула не 
верна.

\item Рассмотрим сигнатуру $(\{D\}, \{ * : D \times D \to D, 1 : D, f : D \to D, g : D \to D, i_1 : D \to D, i_2 : D \to D \})$.
    Рассмотрим следующую теорию в ней:
\begin{align*}
(x * y) * z & = x * (y * z) \\
x * 1 & = x \\
1 * x & = x \\
f(f(x)) & = f(x) \\
g(g(x)) & = g(x) \\
f(g(x)) & = g(f(x)) \\
i_1(f(x)) * g(x) & = 1 \\
f(x) * i_2(g(x)) & = 1
\end{align*}
Какие из следующих утверждений являются теоремами этой теории? Докажите это.
\begin{enumerate}
\item $i_1(x) = i_2(x)$
\item $i_1(x) * x = 1$
\item $f(x) = g(x)$
\item $f(x) = x$
\end{enumerate}
При доказательстве выводимости можно опускать очевидные шаги, такие как применения ассоциативности и аксиом $1 * 
x = x$ и $x * 1 = x$.

\item Рассмотрим сигнатуру $(\{D\}, \{ * : D \times D \to D, + : D \times D \to D, 1 : D, 0 : D, - : D \to D \})$.
    Теория колец с единицей выглядит следующим образом:
\begin{align*}
(x + y) + z & = x + (y + z) \\
x + 0 & = x \\
0 + x & = x \\
x + y & = y + x \\
x + -x & = 0 \\
(x * y) * z & = x * (y * z) \\
x * 1 & = x \\
1 * x & = x \\
x * (y + z) & = (x * y) + (x * z) \\
(y + z) * x & = (y * x) + (z * x)
\end{align*}
Добавим к этой теории следующую аксиому:
\[ x * x = x \]
Докажите, что в этой расширенной теории выводимы следующие формулы:
\begin{enumerate}
\item $x * y = y * x$
\item $x + x = 0$
\begin{align*}
&x + x = x + x + 0 + 0 = x + x + x + x + (-x) + (-x) = (xx + xx) + (xx + xx) + (-x) + (-x) = \\
&= x(x + x) + x(x + x) + (-x) + (-x) = (x + x)(x + x) + (-x) + (-x) = x + x + (-x) + (-x) = \\
&= (x + (-x)) + (x + (-x)) = 0 + 0 = 0
\end{align*}
\end{enumerate}

\end{enumerate}

