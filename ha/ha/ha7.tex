\section*{Классическая логика}

\begin{enumerate}

\item Докажите, что формула $((P \to Q) \to P) \to P$ невыводима, показав, что ее выводимость влечет выводимость исключенного третьего.

\textbf{Решение.}
Предположим, что это правда, тогда правдой является и формула (по лемме об уточнении): $(((S\lor \neg S) \to \bot) \to (S \lor \neg S)) \to (S \lor \neg S)$. (Формула полученной подстановкой $P = (S \lor \neg S), Q = \bot$). 

Заметим, что в этом предположении выводится исключённое третье (обозначим $\varTheta = (((S\lor \neg S) \to \bot) \to (S \lor \neg S)) \to (S \lor \neg S)$):
\begin{center}
	\AxiomC{}
	\UnaryInfC{$\vdash \varTheta$}
	\AxiomC{}
	\UnaryInfC{$ ((S\lor \neg S) \to \bot), S \vdash ((S\lor \neg S) \to \bot) $}
	\AxiomC{}
	\UnaryInfC{$ ((S\lor \neg S) \to \bot), S \vdash S $}
	\UnaryInfC{$ ((S\lor \neg S) \to \bot), S \vdash (S \lor \neg S) $}
	\BinaryInfC{$ ((S\lor \neg S) \to \bot), S \vdash \bot $}
	\UnaryInfC{$ ((S\lor \neg S) \to \bot) \vdash S \to \bot $}
	\UnaryInfC{$ ((S\lor \neg S) \to \bot) \vdash \neg S $}
	\UnaryInfC{$ ((S\lor \neg S) \to \bot) \vdash (S \lor \neg S) $}
	\UnaryInfC{$\vdash ((S\lor \neg S) \to \bot) \to (S \lor \neg S) $}
	\BinaryInfC{$\vdash S \lor \neg S$}
	\DisplayProof
\end{center}

Этого не может быть, значит, мы получили противоречие.

\item Докажите, что в интуиционистской логике следующие правила вывода эквивалентны:
\begin{center}
\AxiomC{}
\RightLabel{, (lem)}
\UnaryInfC{$\Gamma \vdash \varphi \lor \neg \varphi$}
\DisplayProof
\qquad
\AxiomC{}
\RightLabel{, (dne)}
\UnaryInfC{$\Gamma \vdash \neg \neg \varphi \to \varphi$}
\DisplayProof
\end{center}
Для этого достаточно показать, что имея правило (lem), заключение правила (dne) выводимо, и наоборот.

\textbf{Решение.} $lem \Rightarrow dne$. Честно построим дерево вывода.
	\begin{center}
		\AxiomC{}
		\UnaryInfC{$\vdash \varphi \lor \neg \varphi$}
		\AxiomC{}
		\UnaryInfC{$\neg \varphi ,(\neg\varphi \to \bot) \vdash \neg\varphi \to \bot$}
		\AxiomC{}
		\UnaryInfC{$\neg \varphi ,(\neg\varphi \to \bot) \vdash \neg\varphi$}
		\BinaryInfC{$\neg \varphi ,(\neg\varphi \to \bot) \vdash \bot$}
		\UnaryInfC{$\neg \varphi ,(\neg\varphi \to \bot) \vdash \varphi$}
		\UnaryInfC{$\neg \varphi \vdash (\neg\varphi \to \bot) \to \varphi$}
		\AxiomC{}
		\UnaryInfC{$\varphi, (\neg\varphi \to \bot) \vdash \varphi$}
		\UnaryInfC{$\varphi \vdash (\neg\varphi \to \bot) \to \varphi$}
		\TrinaryInfC{$\vdash (\neg\varphi \to \bot) \to \varphi$}
		\UnaryInfC{$\vdash \neg \neg \varphi \to \varphi$}
		\DisplayProof
	\end{center}
	$dne \Rightarrow lem$	
	\begin{center}
		\AxiomC{}
		\UnaryInfC{$\vdash \neg \neg (\varphi \lor \neg \varphi) \to \varphi \lor \neg \varphi$}
		\AxiomC{$\vdash \neg \neg (\varphi \lor \neg \varphi)$}
		\BinaryInfC{$\vdash \varphi \lor \neg \varphi$}
		\DisplayProof
	\end{center}
	
	Заметим, что выражение слева следует из $dne$. Осталось доказать лишь выражение справа. Сделаем это с помощью терма, которому можно приписать тип $(\varphi \lor (\varphi \to \bot)) \to \bot) \to \bot$:
	\begin{equation*}
		\lambda f. f \ (Right \ (\lambda x. f (Left \ x))) 
	\end{equation*}

	

\item Формула $\varphi \land \psi$ доказуема тогда и только тогда когда доказуемы $\varphi$ и $\psi$.
    Верно ли аналогичное утверждение для $\varphi \lor \psi$?
    То есть, правда ли что $\varphi \lor \psi$ доказуема тогда и только тогда когда доказуема либо $\varphi$, либо $\psi$?
    В классической логике это не верно.
    Действительно, $P \lor \neg P$ доказуема, но ни $P$, ни $\neg P$ не доказуемы.
    Докажите, что в интуиционистской логике это свойство выполнено.
    
    Hint: используйте в качестве доказательств формул лямбда термы и примените нормализацию.

\item Докажите в классической логике формулу $\neg (\varphi \land \psi) \to \neg \varphi \lor \neg \psi$
    Напишите лямбда терм, доказывающий эту формулу.
    
    \textbf{Решение.}
    В классический логике нам достаточно перебрать значения переменных $\varphi, \psi$, чтобы доказать формулу.
   
   \begin{longtable}{c|c|c|c|c}
   		$\varphi$ & $\psi$ & $\neg(\varphi \land \psi)$ & $\neg \varphi \lor \neg \psi$ & $\neg(\varphi \land \psi) \to \neg \varphi \lor \neg \psi$  \\ \hline
   		0 & 0 & 1 & 1 & 1 \\ 
   		0 & 1 & 1 & 1 & 1 \\ 
   		1 & 0 & 1 & 1 & 1 \\ 
   		1 & 1 & 0 & 0 & 1 \\ 
   	\end{longtable}
   	
   	Терм, который это доказывает:
   	\begin{align*}
	   	term \ f = case \ lem_{\varphi} \ of \ \{Right \ f' \to Left \ f'; Left \ \varphi' \to Right \ f \ . \ (\psi \to (\varphi', \psi))\}
   	\end{align*}

\item Докажите в интуиционистской логике формулу $(((P \to Q) \lor (Q \to P)) \to P) \to P$.
    Напишите лямбда терм, доказывающий эту формулу.

	\textbf{Решение.}
	
	Формула доказывается, например, таким термом (терм \textit{task5} в файле \textbf{proofs.hs}):
	\begin{equation*}
		\lambda f. f(Right \ (\lambda q. f (Left \ (\lambda p.q))))
	\end{equation*}

\item Реализуйте алгоритм, возвращающей по формуле ее доказательство в классической логике.

\item Реализуйте алгоритм, возвращающей по классическому доказательству формулы $\neg \psi$ ее интуиционистское доказательство.

\end{enumerate}
