\section*{Множества}
\begin{enumerate}

\item Определите множество частичных функций через множество их графиков.

\textbf{Решение.} 

Множество подмножеств $G \subseteq A \times (B \cup \{undefined\})$,таких что для любого $a \in A$ существует 
единственный $b \in (B \cup \{undefined\})$, такой что $(a, b) \in G$.

\item Композиция функций $f : A \to B$ и $g : B \to C$ -- это функция $g \circ f : A \to C$, такая что $(g 
\circ f)(a) = g(f(a))$.
    В хаскелле есть аналог этой операции -- это $g\ .\ f$. Задайте график функции $g \circ f$ через графики 
    $f$ и $g$.

\textbf{Решение.} 

Рассмотрим графики функций $f$ и $g$. Обозначим $G_f \subseteq A \times B$, $G_g \subseteq B \times C$. 
Обозначим $B_f$ множество элементов $b \in B \exists a \in A : (a, b) \in G_f$. А в качестве $B_g$ обозначим 
множество элементов $b\in B : \exists c \in C : (b, c) \in G_g$. Тогда введём $B = B_f \cap B_g$. Тогда в 
качестве графика функции $g \circ f$ можно взять множества пар $(a, c)$ таких, что $\exists b \in B : (a, b) 
\in G_f \land (b, c) \in G_g$.  

\item Докажите, что если $A$ вкладывается в $B$ и $B$ вкладывается в $C$, то $A$ вкладывается в $C$.

\textbf{Решение.} 

$A$ вкладывается в $B$ $\Rightarrow \exists f:A\rightarrow B - $ инъекция.

$B$ вкладывается в $C$ $\Rightarrow \exists g:B\rightarrow C - $ инъекция.

Покажем, что композиция $g \circ f : A\rightarrow C - $ является инъекцией.

Для этого достаточно показать, что если $(g \circ f)(a) = (g \circ f)(a')$ то $a = a'$. Заметим, что 
$g:B\rightarrow C$ - инъекция, это значит, что $f(a) = f(a')$. $f:A\rightarrow C$ - инъекция, значит $a = a'$.

\item Докажите, что если $A$ накрывает $B$ и $B$ накрывает $C$, то $A$ накрывает $C$.

\textbf{Решение.} 

$A$ накрывает в $B$ $\Rightarrow \exists f:A\rightarrow B - $ сюръекция.

$B$ накрывает в $C$ $\Rightarrow \exists g:B\rightarrow C - $ сюръекция.

Покажем, что композиция $g \circ f : A\rightarrow C - $ является сюръекцией.

Для этого достаточно показать, что $\forall c \in C \exists a \in A : (g \circ f)(a) = c$. заметим, что $g$ - 
сюръекция, значит, $\forall c \exists b \in B : g(b) = c$. Теперь, заметим, что $f$ - сюръекция, значит, 
$\forall b \in B \exists a \in A : f(a) = b$. Объединяя эти два утверждения, получим определение сюръекции 
для $(g \circ f)$. $\forall c \in C \exists b \in B : g(b) = c \exists a \in A : f(a) = b \Rightarrow \forall 
c \in C \exists a \in A : g(f(a)) = c$. Значит композиция сюръекций снова сюръекция.

\item Докажите, что следующие свойства верны:
\begin{enumerate}
\item $A$ равномощно $A$.

\textbf{Решение.} 

Существует тождественное отображение $id: A\rightarrow A$, $id(a) = a$. Т.к. это биекция(для $a\neq \hat{a}$, 
они переходят $id(a) = a \neq \hat{a} = id(\hat{a}),$ и $\forall a\in A \exists\hat{a} = a \in A : 
id(\hat{a}) = a$ ), то $A$ равномощно $A$. 

\item Если $A$ равномощно $B$, то $B$ равномощно $A$.

\textbf{Решение.} 

$A$ равномощно $B$ $\Rightarrow$ $\exists \varphi(a):A\rightarrow B$ - биекция $\Leftrightarrow$ $\exists 
\varphi^{-1} : B \rightarrow A$ - обратная функция - биекция $\Rightarrow$ $B$ равномощно $A$ 

\item Если $A$ равномощно $B$ и $B$ равномощно $C$, то $A$ равномощно $C$.

\textbf{Решение.} 

Из такой постановки задачи следуют постановки задач пунктах $3, 4$. В них мы нашли функцию $\varphi = g \circ 
f$, которая является и накрытием и вложением $A$ в $C$, а значит, инъекцией и сюръекцией $\Rightarrow$ 
$\varphi$ является биекцией $\Rightarrow$ $A$ равномощно $C$.

\end{enumerate}

\item Докажите, что множества $A \amalg B \to C$ и $(A \to C) \times (B \to C)$ равномощны.

\textbf{Решение.} 

Приведем биекцию $\varphi : (A \amalg B \to C) \rightarrow ((A \to C) \times (B \to C))$

\begin{equation*}
	\varphi(f) = (a \mapsto f(Left(a)), b \mapsto f(Right(b)))
\end{equation*}

и обратная к ней

\begin{align*}
	& \varphi^{-1}(g) = Left(x) \mapsto \pi_1(g)(x) \\
	& \varphi^{-1}(g) = Right(x) \mapsto \pi_2(g)(x)
\end{align*}

Несложно убедиться, что эти функции взаимно обратные.

Выберем произвольную функцию $f\in A \amalg B \to C$:
\begin{align*}
	& f(Left(x)) = e \\
	& f(Right(x)) = d
\end{align*}

и вычислим значение $\varphi^{-1}(\varphi(f))$:

\begin{align*}
	& \varphi^{-1}(\varphi(f)) = \pi_1(a \mapsto f(Left(a)), b \mapsto 	f(Right(b))) = f(Left(x)) \\
	& \varphi^{-1}(\varphi(f)) = \pi_2(a \mapsto f(Left(a)), b \mapsto 	f(Right(b))) = f(Right(x))
\end{align*}

Так же несложно показать и обратное преобразование:

\begin{equation*}
\begin{array} {lcl}
	\varphi(\varphi^{-1}(g)) = (a \mapsto \varphi^{-1}(g)(Left(a)), b 	\mapsto \varphi^{-1}(g)(Right(b))) = 
	\\
	= (a \mapsto \pi_1(g)(Left(a)), b \mapsto \pi_2(g)(Right(b))) = (\pi_1(g), \pi_2(g)) = g
\end{array}
\end{equation*}

Таким образом, $\varphi$ - биекция, и множества $A \amalg B \to C$ и $(A \to C) \times (B \to C)$ равномощны.

\item Докажите, что множества $A \times B \to C$ и $A \to (B \to C)$ равномощны.

\textbf{Решение.} 

По аналогии с предыдущим заданием придумаем биекцию $\varphi : (A \times B \to C)\to (A \to (B \to C))$.

\begin{equation*}
	\varphi(f) = x \mapsto (y \mapsto f(x, y))
\end{equation*}

Можно заметить, что получили каррирование функции $f \in A \times B \to C$.

Обратная функция к $\varphi(f)$:

\begin{equation*}
	\varphi^{-1}(g) = p \mapsto g(\pi_1(p))(\pi_2(p))
\end{equation*}

Таким образом, $\varphi$ - биекция, и множества $A \times B \to C$ и $A \to (B \to C)$ равномощны.

\item Докажите, что $|\mathcal{P}(A)| = 2^{|A|}$ и $|A \to B| = |B|^{|A|}$.

\textbf{Решение.} 

Очевидно, первое равенство является прямым следствием второго, т.к. $\mathcal{P}(A)$ имеет 
взаимно-однозначное соответствие с отображениями $A \to \Omega$, где $\Omega = \{ \top, \bot \}$, значит 
$|\mathcal{P}(A)| = |A \to \Omega| = |\Omega|^{|A|} = 2^{|A|}$.

Докажем второе равенство.

Для этого достаточно найти биекцию $A \to B$ на множество 

$\{0, 1, ..., |B|^{|A|} - 1 \}$.

Раз уж $A$ и $B$ конечные, то из элементы можно пронумеровать номерами $\{0, 1, 2,..., |A|\}$ и $\{0, 1, 
2,..., |B|\}$ соответственно. Теперь, если записать последовательно $A$ образов элементов множества $A$ для 
которой $f \in A \to B$, то получим последовательность $f(a_{|A|}), f(a_{|A| - 1}), f(a_{|A| - 2}), .., 
f(a_2), f(a_1)$ элементов из $B$. Их можно отобразить на множество $\{0, 1, 2,..., |B|\}$ Получим число в 
системе счисления с основанием равным $|B|$ и $|A|$ разрядами. Переведя это число по соответствующим правилам 
в десятичную систему исчиления получим значение из интервала $[0; |B|^{|A|} - 1]$. Так же есть возможность 
совершить обратное преобразование (последовательное взятие остатка по модулю $|B|$ и целочисленное деление на 
значение $|B|$). Таким образом, требуемая биекция найдена, следовательно $|A \to B| = |B|^{|A|}$.

\end{enumerate}
