\section{Индукция}
\begin{enumerate}

\item Напишите нерекурсивное определение функции \[ f(n) = \sum_{i < n} (f(i) + 1) \]

Докажите, используя (обобщенный) принцип индукции, равенство этих двух функций.

\textbf{Решение.} 

Это выражение образует последовательность вида: $0, 1, 3, 7, 15,...$ .Нерекурсивное определение функции: 
\begin{equation*}
f(n) = 2^n - 1
\end{equation*}

Докажем это равенство используя обобщенный принцип индукции.

Сначала заметим, что $f(0) = 2^0 - 1 = 1 - 1 = 0$. Верно.

Докажем индукционный переход. Пусть это верно для $f(n) = 2^n$. Покажем, что верно и для $f(S(n)) = f(n + 1)$.
\begin{equation*}
f(n + 1) = \sum_{i < n + 1} (f(i) + 1) = \sum_{i<n}(2^i) = 2^{n + 1} - 1.
\end{equation*}
Доказано.

\item Докажите, что принцип зависимой рекурсии эквивалентен принципам рекурсии и индукции. Hint: Чтобы 
доказать, что принцип индукции следует из принципа зависимой рекурсии, возьмите в качестве $B$ следующую 
коллекцию: $B(n) = \{ * \}$, если верно $P(n)$, иначе $B(n) = \varnothing$.
    
\textbf{Решение.}

\begin{itemize}
	\item Зависимая рекурсия $\Rightarrow$ рекурсия.
	
	Очевидно, понятие зависимой рекурсии обобщает понятие обычной рекурсии:
	
	Выберем $B(0) = B(1) = B(2) = \ldots = B(n) = \ldots = B$. Получим обычную рекурсию.
	
	\item Зависимая рекурсия $\Rightarrow$ индукция.
	
	Воспользуемся подсказкой, тогда чтобы проверить, что некоторое утверждение $P(n)$ выполняется для 	
	произвольных $n$ достаточно, чтобы $\bigcap\limits_{n\in \mathbb{N}} B(n) \neq \varnothing$. То есть 
	достаточно показать, что $P(0)\in \{*\}$, а так же $P(n) \in \{*\} \Rightarrow P(S(n)) \in \{*\}$.
	
	\item Рекурсия $\wedge$ индукция $\Rightarrow$ зависимая рекурсия.
	
	В определении рекурсии выберем $B = \bigcup\limits_{n\in \mathbb{N}} B(n)$.
	
	А в индукции в качестве $P(n)$ можно выбрать утверждение $f(n) \in B(n)$. В результате, получим принцип 
	зависимой рекурсии.
\end{itemize}

\item Приведите контрпримеры, показывающие, что отдельно ни принципа рекурсии, ни принципа индукции не 
достаточно, чтобы гарантировать уникальность натуральных чисел. То есть нужно привести примеры множеств 
$\mathbb{N}_i$ вместе с $0_i \in \mathbb{N}_i$, $S_i : \mathbb{N}_i \to \mathbb{N}_i$, где $i \in \{ 1, 2 
\}$, таких что $\mathbb{N}_1$ удовлетворяет принципу рекурсии, $\mathbb{N}_2$ удовлетворяет принципу 
индукции, но они не равномощны $\mathbb{N}$.
    
\textbf{Решение.} Пусть $\mathbb{N}_2 = \{0,1,2\}, 0_2 = 0, S_2(0) = 1, S_2(1) = 2, S_2(2) = 1$. Эта 
конструкция удовлетворяет принципу индукции. Но задать функцию с помощью принципа рекурсии не позволяет в 
силу неоднозначности рекурсивного вызова для $f(1)$. $\mathbb{N}_2$ конечно $\Rightarrow$ оно не равномощно 
множеству натуральных чисел. 

$\mathbb{N}_1 = \{0,1,2\}, 0_1 = 1, S_1(0) = 1, S_1(1) = 2, S_1(2) = 1$. Эта конструкция позволяет задать 
рекурсивные функции. Но доказывать по индукции не позволяет, т.к. утверждение $P$ для $0\in \mathbb{N}_1$ 
может быть как истинным так и ложным, и это не повлияет на выполнение предиката на всех элементах множества 
$\mathbb{N}_1$. $\mathbb{N}_1$ конечно $\Rightarrow$ оно не равномощно множеству натуральных чисел. 

\item Задача снята.

\item Сформулируйте принципы рекурсии, индукции и зависимой рекурсии для множества $List(A)$.

\textbf{Решение.} Хоть и первые два принципа следуют из третьего, приведём все три.

\begin{itemize}
	\item Рекурсия.
	
	Для задания функций $f:List(A) \rightarrow B$ достаточно задать следующие данные:
	\begin{align*}
		&f(nil) = b \\
		&f(cons(a, xs))= e 
	\end{align*}
	где $b \in B$, $e$ - выражение, в котором может быть вызов $f(xs)$, и которое задаёт элемент $B$.
	\item Индукция. Позволяет доказывать, что некоторый предикат $P(xs)$ выполняется на всех элементах $xs 
	\in List(A)$.
	
	Для этого достаточно показать, что верно $P(nil)$;
	
	А так же, $\forall xs \in List(A) : \left[ \left\{ P(xs) = \top \right\} \Rightarrow \left\{ \forall a 
	\in A : P(cons(a, xs)) = \top \right\} \right]$
	
	\item Зависимая рекурсия.
	
	Для задания функций $f:List(A) \rightarrow \Pi(xs \in List(A))B(xs)$ достаточно задать 
	следующие данные:
	\begin{align*}
	&f(nil) = b \\
	&f(cons(a, xs))= e 
	\end{align*}
	где $b \in B(nil)$, $e$ - выражение, в котором может быть вызов $f(xs) \in B(xs)$, и которое задаёт 
	элемент $(cons(a, xs) \in B(cons(a, xs))$.
\end{itemize}

\item Опишите индуктивным образом предикат на $\mathbb{N}$, задающий нечётные числа.

\textbf{Решение.}

\begin{center}
	\AxiomC{}
	\UnaryInfC{$1\ is\ odd$}
	\DisplayProof
	\qquad
	\AxiomC{$n\ is\ odd$}
	\UnaryInfC{$n + 2\ is\ odd$}
	\DisplayProof
\end{center}

\end{enumerate}

