\section*{Индукция}
\begin{enumerate}

\item Напишите нерекурсивное определение функции
\[ f(n) = \sum_{i < n} (f(i) + 1) \]

Докажите, используя (обобщенный) принцип индукции, равенство этих двух функций.


\item Докажите, что принцип зависимой рекурсии эквивалентен принципам рекурсии и индукции.
    Hint: Чтобы доказать, что принцип индукции следует из принципа зависимой рекурсии, возьмите в качестве $B$ следующую коллекцию:
    $B(n) = \{ * \}$, если верно $P(n)$, иначе $B(n) = \varnothing$.

\item Приведите контрпримеры, показывающие, что отдельно ни принципа рекурсии, ни принципа индукции не достаточно, чтобы гарантировать уникальность натуральных чисел.
    То есть нужно привести примеры множеств $\mathbb{N}_i$ вместе с $0_i \in \mathbb{N}_i$, $S_i : \mathbb{N}_i \to \mathbb{N}_i$, где $i \in \{ 1, 2 \}$,
    таких что $\mathbb{N}_1$ удовлетворяет принципу рекурсии, $\mathbb{N}_2$ удовлетворяет принципу индукции, но они не равномощны $\mathbb{N}$.

\item Пусть $\mathbb{N}'$ -- некоторое множество вместе с $0' \in \mathbb{N}'$ и $S' : \mathbb{N}' \to \mathbb{N}'$.
    Тогда принцип $PM$ для $\mathbb{N}'$ говорит, что для любых $n,m \in \mathbb{N}$ если $S'^n(0') = S'^m(0')$, то $n = m$.
\begin{itemize}
\item Докажите, что принцип рекурсии для $\mathbb{N}'$ эквивалентен $PM$.
\item Обратите внимание, что для доказательства этого факта нужно использовать принцип исключенного третьего. Укажите явно место, где оно необходимо.
\item Какое дополнительное предположение о $\mathbb{N}'$ нужно сделать, чтобы это доказательство работало без исключенного третьего?
\end{itemize}

\item Сформулируйте принципы рекурсии, индукции и зависимой рекурсии для множества $List(A)$.

\item Опишите индуктивным образом предикат на $\mathbb{N}$, задающий нечетные числа.

\end{enumerate}

