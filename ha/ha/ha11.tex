\section*{Теория множеств}

\begin{enumerate}

\item Докажите, что отношение $\subseteq$ является частичным порядком. Какие аксиомы при этом необходимо 
использовать?

\textbf{Решение.} Покажем рефлексивность, транзитивность и антисимметричность.
\begin{itemize}
	\item Рефлексивность. $a \subseteq a$. Нужно $\forall z \ (z \in a \to z \in a)$. Это утверждение верное.
	\item Транзитивность. $a \subseteq b \land b \subseteq c \to a \subseteq c$. Покажем, что $\forall z \ z \in a \to z \in c$. 
	\begin{itemize}
		\item $\forall z \ (z \in a \to z \in b)$. Воспользуемся $b \subseteq c$: $(\forall z \ (z \in a \to z \in b 
		\to z \in c)) \to \forall z \ (z \in a \to z \in c)$.
	\end{itemize}
	\item Антисимметричность. $\forall x\forall y (x \subseteq y \land y \subseteq x \to x = y)$. Это аксиома 
	экстенсиональности.
\end{itemize}
\item Мы говорили, что $\in$-индукция влечет тот факт, что не существует бесконечной последовательности множеств 
$\ldots \in x_2 \in x_1 \in x_0$.
    Докажите при помощи $\in$-индукции следующие частные случаи этого утверждения:
\begin{enumerate}
\item Отношение $\in$ иррефлексивно.
\item Не существует $x$ и $y$, таких что $x \in y$ и $y \in x$.
\end{enumerate} 

\item Докажите следующие свойства натуральных чисел:
\begin{enumerate}
\item $\forall x \in \mathbb{N} (x = 0 \lor \exists y \in \mathbb{N} (x = S(y)))$.
\item Используя предыдущий пункт и следующий принцип индукции
\[ 
(\forall x \in \mathbb{N}\ (\forall y < x\ \varphi(y)) \to \varphi(x)) \to \forall x \in \mathbb{N}\ \varphi(x), 
\]
докажите обычный принцип индукции для натуральных чисел:
\[ \varphi(0) \land (\forall y \in \mathbb{N}\ \varphi(y) \to \varphi(S(y))) \to \forall x \in \mathbb{N}\ 
\varphi(x), \]
\end{enumerate}

\item Докажите следующие свойства натуральных чисел (hint: каждый следующий пункт следует из предыдущего):
\begin{enumerate}
\item Если $S(x) < S(y)$, то $x < y$.

\textbf{Решение.}

Нужно показать, что $\left[x \cup \{x\} \in y \cup \{y\}\right] \to x \in y$. Рассмотрим возможные случаи:
\begin{itemize}
	\item $x \cup \{x\} = y$. Из этого сразу следует, что $x \in y$.
	\item $x \cup \{x\} \in y$. Теперь воспользуемся тем, что нам известна структура $y$. Т.е. если $x \cup \{x\} 
	\in	y$, то $x \in y$.
\end{itemize}
\item $x < y$ тогда и только тогда, когда $S(x) \leq y$.

\textbf{Решение.}
\begin{itemize}
	\item $\Leftarrow$ доказано на лекции.
	\item $\Rightarrow$ Нужно показать, что $x \in y \to x \cup \{x\} \subseteq y$. Или, что то же самое, $x \in 
	y \to \forall z \ (z \in x \cup \{x\} \to z \in y)$. При $z = x$ это верно, т.к. $x < y$. Остался случай, 
	когда $z \in x$. Заметим, что $x < y \to x \leq y \to \forall z (z \in x \to z \in y)$.
\end{itemize}

\item Отношение $\in$ транзитивно на элементах $\mathbb{N}$.
    Можно показать более общее утверждение: если $x \in y \in z \in \mathbb{N}$, то $x \in z$.
\item Если $x \in y$ и $y \in \mathbb{N}$, то $x \in \mathbb{N}$.
\end{enumerate}

\end{enumerate}
