\section*{Теория множеств}

\begin{enumerate}

\item Докажите, что отношение $\subseteq$ является частичным порядком. Какие аксиомы при этом необходимо 
использовать?

\item Мы говорили, что $\in$-индукция влечет тот факт, что не существует бесконечной последовательности множеств 
$\ldots \in x_2 \in x_1 \in x_0$.
    Докажите при помощи $\in$-индукции следующие частные случаи этого утверждения:
\begin{enumerate}
\item Отношение $\in$ иррефлексивно.
\item Не существует $x$ и $y$, таких что $x \in y$ и $y \in x$.
\end{enumerate}

\item Докажите следующие свойства натуральных чисел:
\begin{enumerate}
\item $\forall x \in \mathbb{N} (x = 0 \lor \exists y \in \mathbb{N} (x = S(y)))$.
\item Используя предыдущий пункт и следующий принцип индукции
\[ (\forall x \in \mathbb{N}\ (\forall y < x\ \varphi(y)) \to \varphi(x)) \to \forall x \in \mathbb{N}\ \varphi(x), 
\]
докажите обычный принцип индукции для натуральных чисел:
\[ \varphi(0) \land (\forall y \in \mathbb{N}\ \varphi(y) \to \varphi(S(y))) \to \forall x \in \mathbb{N}\ 
\varphi(x), \]
\end{enumerate}

\item Докажите следующие свойства натуральных чисел (hint: каждый следующий пункт следует из предыдущего):
\begin{enumerate}
\item Если $S(x) < S(y)$, то $x < y$.
\item $x < y$ тогда и только тогда, когда $S(x) \leq y$.
\item Отношение $\in$ транзитивно на элементах $\mathbb{N}$.
    Можно показать более общее утверждение: если $x \in y \in z \in \mathbb{N}$, то $x \in z$.
\item Если $x \in y$ и $y \in \mathbb{N}$, то $x \in \mathbb{N}$.
\end{enumerate}

\end{enumerate}

