

Пусть $M$ -- частично упорядоченное множество и $x,y \in M$.
Тогда \emph{супремум} $x$ и $y$ -- это элемент $x \lor y \in M$, удовлетворяющий следующим условиям:
\begin{itemize}
\item $x \leq x \lor y$.
\item $y \leq x \lor y$.
\item Для любого $\gamma$ если $x \leq \gamma$ и $y \leq \gamma$, то $x \lor y \leq \gamma$.
\end{itemize}
Супремум двух элементов может не существовать, но если он существует, то он уникален.

\emph{Инфимум} $x$ и $y$ -- это элемент $x \land y \in M$, удовлетворяющий следующим условиям:
\begin{itemize}
\item $x \land y \leq x$.
\item $x \land y \leq y$.
\item Для любого $\gamma$ если $\gamma \leq x$ и $\gamma \leq y$, то $\gamma \leq x \land y$.
\end{itemize}
Инфимум двух элементов может не существовать, но если он существует, то он уникален.

Эти условия в определениях супремума и инфимума можно записать в следующем виде (это те же самые условия, но в 
такой форме их может быть проще воспринимать):
\begin{center}
\AxiomC{}
\UnaryInfC{$x \leq x \lor y$}
\DisplayProof
\qquad
\AxiomC{}
\UnaryInfC{$y \leq x \lor y$}
\DisplayProof
\qquad
\AxiomC{$x \leq \gamma$}
\AxiomC{$y \leq \gamma$}
\BinaryInfC{$x \lor y \leq \gamma$}
\DisplayProof
\end{center}

\begin{center}
\AxiomC{}
\UnaryInfC{$x \land y \leq x$}
\DisplayProof
\qquad
\AxiomC{}
\UnaryInfC{$x \land y \leq y$}
\DisplayProof
\qquad
\AxiomC{$\gamma \leq x$}
\AxiomC{$\gamma \leq y$}
\BinaryInfC{$\gamma \leq x \land y$}
\DisplayProof
\end{center}

\begin{enumerate}

\item Закончите доказательство того, что интерпретация логики с $\land$ и $\lor$ в дистрибутивных решетках 
корректна.
    То есть нужно доказать, что если $\gamma \leq \varphi \lor \psi$, $\gamma \land \varphi \leq \chi$ и $\gamma \land \psi \leq \chi$, то $\gamma \leq \chi$.
    
    \textit{Решена в классе}

\item Докажите, что в любой решетке для любых элементов $x$, $\varphi$ и $\psi$ следующие утверждения эквивалентны:
\begin{enumerate}
\item Для любого $\gamma$ если $\gamma \leq x$, то $\gamma \land \varphi \leq \psi$.
\item Для любого $\gamma$ если $\gamma \leq x$ и $\gamma \leq \varphi$, то $\gamma \leq \psi$.
\item $x \land \varphi \leq \psi$.
\end{enumerate}

\textit{Решена в классе}

\item Пусть $\varphi$, $\psi$, $x$ и $x'$ -- элементы решетки. Допустим в ней верны следующие свойства:
\begin{enumerate}
\item Для любого $\gamma$
    \[ \gamma \leq x \Leftrightarrow \gamma \land \varphi \leq \psi \]
\item Для любого $\gamma$
    \[ \gamma \leq x' \Leftrightarrow \gamma \land \varphi \leq \psi \]
\end{enumerate}
Докажите, что тогда $x = x'$.

\textbf{Решение.} 

Подставим в $(a)$ вместо $\gamma$ элемент $x$. Очевидно, $x \leq x$. Значит, верно, что $x \land \varphi \leq 
\psi$. Заметим, что это совпадает с левой частью $(b)$. Значит, $x \leq x'$.

Можно провести аналогичные рассуждения для $x'$ и $(b)$, получив $x' \leq x$. 
\begin{equation*}
	(x \leq x') \land (x' \leq x) \Rightarrow (x = x')
\end{equation*}

\item Покажите, что в любой алгебре Гейтинга $M$ верны следующие свойства:
\begin{enumerate}
\item В $M$ существует наибольший элемент, то есть элемент $\top$, удовлетворяющий условию, что $x \leq \top$ для 
любого $x$.
\item Для любых $\varphi, \psi \in M$ верно $\varphi \leq \psi \Leftrightarrow (\varphi \to \psi) = \top$
\end{enumerate}

\textit{Решена в классе}

\item Докажите, что любая алгебра Гейтинга дистрибутивна.
    Hint: Нужно отдельно доказать, что
    $(\varphi \land \psi) \lor (\varphi \land \chi) \leq \varphi \land (\psi \lor \chi)$ и
    $\varphi \land (\psi \lor \chi) \leq (\varphi \land \psi) \lor (\varphi \land \chi)$.
    Первое свойство верно в любой решетке, при доказательстве второго используется, что решетка ялвяется алгеброй Гейтинга.
    
    \textbf{Решение.}
    
    Сначала покажем, что $(\varphi \land \psi) \lor (\varphi \land \chi) \leq \varphi \land (\psi \lor \chi)$.
    \begin{center}
    	\AxiomC{}
    	\UnaryInfC{$\varphi \land \psi \leq \varphi$}
    	\AxiomC{}
    	\UnaryInfC{$\varphi \land \psi \leq \psi \leq \psi \lor \chi$}
    	\UnaryInfC{$\varphi \land \psi \leq \psi \lor \chi$}
    	\BinaryInfC{$\varphi \land \psi \leq \varphi \land (\psi \lor \chi)$}
    	\AxiomC{}
    	\UnaryInfC{$\varphi \land \chi \leq \varphi$}
    	\AxiomC{}
    	\UnaryInfC{$\varphi \land \chi \leq \chi \leq \psi \lor \chi$}
    	\UnaryInfC{$\varphi \land \chi \leq \psi \lor \chi$}
    	\BinaryInfC{$\varphi \land \chi \leq \varphi \land (\psi \lor \chi)$}
    	\BinaryInfC{$(\varphi \land \psi) \lor (\varphi \land \chi) \leq \varphi \land (\psi \lor \chi)$}
    	\DisplayProof
    \end{center}
    
    И теперь обратно, $\varphi \land (\psi \lor \chi) \leq (\varphi \land \psi) \lor (\varphi \land \chi)$.
    

\item Докажите, что в любом частично упорядоченном множестве следующие условия эквивалентны:
\begin{enumerate}
\item Для любого множества $S$ ее элементов существует их супремум $\bigvee S$.
\item Для любого множества $S$ ее элементов существует их инфимум $\bigwedge S$.
\end{enumerate}

	\textbf{Решение.}
	
	$(a) \Rightarrow (b)$. Существует супремум $\Rightarrow$ соответствующая решётка полна $\Rightarrow$ в неё существует наименьший элемент $\bot = \bigwedge \varnothing$
	
	$(b) \Rightarrow (a)$. 

\item Докажите, что следующие формулы не выводимы в пропозициональной логике:
\begin{enumerate}
\item $(P \to Q) \to \neg P \lor Q$.

	\textbf{Решение.}
	
	Выберем $P = (0, 2), Q = (1, 3)$. Теперь 
	\begin{align*}
		P \to Q &= (-\infty; 0) \cup (1; +\infty) \\
		\neg P \lor Q &= (-\infty; 0) \cup (1; +\infty) \\
		(P \to Q) \to (\neg P \lor Q) &= (-\infty; 0) \cup (0, 1) \cup (1, +\infty) \neq \mathbb{R}
	\end{align*}
\item $\neg (P \land Q) \to \neg P \lor \neg Q$.
	\textbf{Решение.}
	
	Выберем $P = (0, 3), Q = (1, 2)$. Теперь
	\begin{align*}
		\neg (P \land Q) &= (-\infty; 1) \cup (2, +\infty) \\
		\neg P \lor \neg Q &= (-\infty; 1) \cup (2, +\infty) \\
		\neg (P \land Q) \to \neg P \lor \neg Q &= (-\infty; 1) \cup (1, 2) \cup (2, +\infty) \neq \mathbb{R}
	\end{align*}

\end{enumerate}

\end{enumerate}

