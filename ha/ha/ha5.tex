\section*{Пропозициональная логика}

Множество типов лямбда исчисления:
\begin{center}
\AxiomC{}
\RightLabel{, $P \in Var$}
\UnaryInfC{$P \in Type$}
\DisplayProof
\qquad
\AxiomC{}
\UnaryInfC{$\bot \in Type$}
\DisplayProof
\qquad
\AxiomC{$\varphi \in Type$}
\AxiomC{$\psi \in Type$}
\BinaryInfC{$\varphi \to \psi \in Type$}
\DisplayProof
\end{center}

\begin{center}
\AxiomC{$\varphi \in Type$}
\AxiomC{$\psi \in Type$}
\BinaryInfC{$\varphi \times \psi \in Type$}
\DisplayProof
\qquad
\AxiomC{$\varphi \in Type$}
\AxiomC{$\psi \in Type$}
\BinaryInfC{$\varphi \amalg \psi \in Type$}
\DisplayProof
\end{center}

Множество предтермов лямбда исчисления (мы используем разные множества для переменных в типах и переменных в термах):
\begin{center}
\AxiomC{}
\RightLabel{, $x \in Var'$}
\UnaryInfC{$x \in Term$}
\DisplayProof
\qquad
\AxiomC{$t \in Term$}
\RightLabel{, $\varphi \in Type$}
\UnaryInfC{$absurd_\varphi\,t \in Term$}
\DisplayProof
\qquad
\end{center}

\begin{center}
\AxiomC{$t \in Term$}
\RightLabel{, $x \in Var'$}
\UnaryInfC{$\lambda x.\,t \in Term$}
\DisplayProof
\qquad
\AxiomC{$t \in Term$}
\AxiomC{$t' \in Term$}
\BinaryInfC{$t\,t' \in Term$}
\DisplayProof
\end{center}

\begin{center}
\AxiomC{$a \in Term$}
\AxiomC{$b \in Term$}
\BinaryInfC{$(a,b) \in Term$}
\DisplayProof
\qquad
\AxiomC{$t \in Term$}
\UnaryInfC{$fst\,t \in Term$}
\DisplayProof
\qquad
\AxiomC{$t \in Term$}
\UnaryInfC{$snd\,t \in Term$}
\DisplayProof
\end{center}

\begin{center}
\AxiomC{$t \in Term$}
\UnaryInfC{$Left\,t \in Term$}
\DisplayProof
\qquad
\AxiomC{$t \in Term$}
\UnaryInfC{$Right\,t \in Term$}
\DisplayProof
\end{center}

\begin{center}
\AxiomC{$e \in Term$}
\AxiomC{$a \in Term$}
\AxiomC{$b \in Term$}
\RightLabel{, $x,y \in Var'$}
\TrinaryInfC{$\mathbf{case}\,e\,\mathbf{of}\,\{\,Left(x) \to a; Right(y) \to b\,\} \in Term$}
\DisplayProof
\end{center}

Правила типизации:
\begin{center}
\AxiomC{}
\RightLabel{, $(x : \varphi) \in \Gamma$}
\UnaryInfC{$\Gamma \vdash x : \varphi$}
\DisplayProof
\qquad
\AxiomC{$\Gamma \vdash b : \bot$}
\UnaryInfC{$\Gamma \vdash absurd_\varphi\,b : \varphi$}
\DisplayProof
\qquad
\end{center}

\begin{center}
\AxiomC{$\Gamma, x : \varphi \vdash b : \psi$}
\UnaryInfC{$\Gamma \vdash \lambda x.\,b : \varphi \to \psi$}
\DisplayProof
\qquad
\AxiomC{$\Gamma \vdash f : \varphi \to \psi$}
\AxiomC{$\Gamma \vdash a : \varphi$}
\BinaryInfC{$\Gamma \vdash f\,a : \psi$}
\DisplayProof
\end{center}

\begin{center}
\AxiomC{$\Gamma \vdash a : \varphi$}
\AxiomC{$\Gamma \vdash b : \psi$}
\BinaryInfC{$\Gamma \vdash (a,b) : \varphi \times \psi$}
\DisplayProof
\qquad
\AxiomC{$\Gamma \vdash p : \varphi \times \psi$}
\UnaryInfC{$\Gamma \vdash fst\,p : \varphi$}
\DisplayProof
\qquad
\AxiomC{$\Gamma \vdash p : \varphi \times \psi$}
\UnaryInfC{$\Gamma \vdash snd\,t : \psi$}
\DisplayProof
\end{center}

\begin{center}
\AxiomC{$\Gamma \vdash a : \varphi$}
\UnaryInfC{$\Gamma \vdash Left\,a : \varphi \amalg \psi$}
\DisplayProof
\qquad
\AxiomC{$\Gamma \vdash b : \psi$}
\UnaryInfC{$\Gamma \vdash Right\,b : \varphi \amalg \psi$}
\DisplayProof
\end{center}

\begin{center}
\AxiomC{$\Gamma \vdash e : \varphi \amalg \psi$}
\AxiomC{$\Gamma, x : \varphi \vdash a : \chi$}
\AxiomC{$\Gamma, y : \psi \vdash b : \chi$}
\TrinaryInfC{$\Gamma \vdash \mathbf{case}\,e\,\mathbf{of}\,\{\,Left(x) \to a; Right(y) \to b\,\} : \chi$}
\DisplayProof
\end{center}

А теперь задания:

\begin{enumerate}

\item Между правилами вывода логики и конструкциями в лямбда исчислении существует естественная биекция.
    Например, $\to\!\!I$ соответствует абстракции, а $\to\!\!E$ соответствует аппликации.
    Запишите эту биекцию для остальных правил и конструкций.
    
\textbf{Решение.} 

\begin{align*}
	\land I &\Leftrightarrow (,) \\
	\land E_1 &\Leftrightarrow snd \\
	\land E_2 &\Leftrightarrow fst \\
	\lor I_1 &\Leftrightarrow Left \\
	\lor I_2 &\Leftrightarrow Right \\
	\lor E &\Leftrightarrow case .. of
\end{align*}


\item Приведите для следующих теорем деревья вывода и термы, доказывающие их:

Тут только деревья вывода, все термы в \textit{\textbf{ha5.hs}}
\begin{enumerate}
\item $P \to P$

\textbf{Решение.}
\begin{center}
	\AxiomC{}
	\UnaryInfC{$P \vdash P$}
	\RightLabel{ $\to I$}
	\UnaryInfC{$\vdash P \to P$}
	\DisplayProof
\end{center}

\item $P \to (P \to Q) \to Q$

\textbf{Решение.}
\begin{center}
	\AxiomC{}
	\UnaryInfC{$P , (P \to Q) \vdash P$}
	\RightLabel{ $\to I$}
	\UnaryInfC{$P \vdash (P \to Q) \to P$}
	\RightLabel{ $\to I$}
	\UnaryInfC{$\vdash P \to (P \to Q) \to P$}
	\DisplayProof
\end{center}
\item $P \land Q \to P \lor Q$
\begin{center}
	\AxiomC{}
	\UnaryInfC{$ P \land Q \vdash P \land Q $}
	\RightLabel{ $ \land E_1 $}
	\UnaryInfC{$ P \land Q \vdash P $}
	\RightLabel{ $ \lor I_1 $}
	\UnaryInfC{$ P \land Q \vdash P \lor Q $}
	\RightLabel{ $\to I$}
	\UnaryInfC{$ \vdash P \land Q \to P \lor Q$}
	\DisplayProof
\end{center}
\item $(P \lor Q) \land R \to (P \land R) \lor (Q \land R)$
\begin{center}
\AxiomC{}
\UnaryInfC{$\Gamma \vdash (P \lor Q) \land R $}
\RightLabel{$\land E_1$}
\UnaryInfC{$\Gamma \vdash P \lor Q$}
\AxiomC{}
\UnaryInfC{$\Gamma, P \vdash P $}
\AxiomC{}
\UnaryInfC{$\Gamma \vdash (P \lor Q) \land R $}
\RightLabel{ $\land E_2$}
\UnaryInfC{$\Gamma, Q \vdash R $}
\RightLabel{ $\land I$}
\BinaryInfC{$\Gamma, P \vdash P \land R $}
\RightLabel{ $\lor I_1$}
\UnaryInfC{$\Gamma, P \vdash (P \land R) \lor (Q \land R) $}
\AxiomC{}
\UnaryInfC{$\Gamma, Q \vdash Q $}
\AxiomC{}
\UnaryInfC{$\Gamma \vdash (P \lor Q) \land R $}
\RightLabel{ $\land E_2$}
\UnaryInfC{$\Gamma, Q \vdash R $}
\RightLabel{ $\land I$}
\BinaryInfC{$\Gamma, Q \vdash Q \land R $}
\RightLabel{ $\lor I_2$}
\UnaryInfC{$\Gamma, Q \vdash (P \land R) \lor (Q \land R) $}
\RightLabel{ $\lor E$}
\TrinaryInfC{$\Gamma \vdash (P \land R) \lor (Q \land R)$}
\RightLabel{ $\to I$}
\UnaryInfC{$\vdash (P \lor Q) \land R \to (P \land R) \lor (Q \land R)$}
\DisplayProof
\end{center}

Где $\Gamma = (P \lor Q) \land R$
\end{enumerate}

\item Приведите для следующих теорем доказывающие их термы:

\textbf{Решение}.

Для всех пунктов см. \textit{\textbf{ha5.hs}}
\begin{enumerate}
\item $(P \land R) \lor (Q \land R) \to (P \lor Q) \land R$
\item $(P \lor Q) \lor R \to P \lor (Q \lor R)$
\item $((((P \to Q) \to P) \to P) \to Q) \to Q$
\end{enumerate}

\item Добавим в нашей логике новую связку $\leftrightarrow$, удовлетворяющую следующим условиям:
\begin{align*}
\top \leftrightarrow \top & = \top \\
\top \leftrightarrow \bot & = \bot \\
\bot \leftrightarrow \top & = \bot \\
\bot \leftrightarrow \bot & = \top
\end{align*}
\begin{enumerate}
\item Опишите правила введения и элиминации для этой связки.
    Они не должны использовать никакие другие связки.
    
    \textbf{Решение.}
\begin{center}
	\AxiomC{$\Gamma \vdash \varphi \leftrightarrow \psi$}
	\AxiomC{$\Gamma \vdash \varphi$}
	\RightLabel{ $\leftrightarrow E_1$}
	\BinaryInfC{$\Gamma \vdash \psi$}
	\DisplayProof
	\qquad
	\AxiomC{$\Gamma \vdash \varphi \leftrightarrow \psi$}
	\AxiomC{$\Gamma \vdash \psi$}
	\RightLabel{ $\leftrightarrow E_1$}
	\BinaryInfC{$\Gamma \vdash \varphi$}
	\DisplayProof
\end{center}

\begin{center}
	\AxiomC{$\Gamma, \varphi \vdash \psi$}
	\AxiomC{$\Gamma, \psi \vdash \varphi$}
	\RightLabel{ $\leftrightarrow I$}
	\BinaryInfC{$\Gamma \vdash \varphi \leftrightarrow \psi$}
	\DisplayProof
\end{center}
    
    
    
    
\item Опишите аналогичные конструкции и правила типизации для них в лямбда исчислении.
\item Приведите терм, доказывающий формулу $(P \lor Q \to R) \leftrightarrow (P \to R) \land (Q \to R)$.
\end{enumerate}

\end{enumerate}
