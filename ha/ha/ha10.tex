\section*{Расшикрения арифметики}

\begin{enumerate}

\item Докажите, что $L_1^{lem}$ является консервативным расширением $L_0^{lem}$.

\textbf{Решение.} 

Достаточно привести эквивалентную формулу в классической логике для выражения с конъюнкцией $\varphi \land \psi$. Можно убедиться (перебрав значения все значения $\varphi, \psi$ в булевой интерпретации $\{\bot,\top\}$), что конъюнкции соответствует формула:
\begin{equation*}
	\neg(\varphi \to \neg \psi)
\end{equation*}

\item Если в $PRCPA$ можно определять функции при помощи паттерн матчинга на нескольких аргументах сразу, то мы легко можем определить функции $min$ и $max$:
\begin{align*}
& min : \mathbb{N} \times \mathbb{N} \to \mathbb{N} \\
& min(0,y) = 0 \\
& min(S(x),0) = 0 \\
& min(S(x),S(y)) = S(min(x,y)) \\
& \\
& max : \mathbb{N} \times \mathbb{N} \to \mathbb{N} \\
& max(0,y) = y \\
& max(S(x),S(y)) = S(max(x,y))
\end{align*}
\begin{enumerate}
\item Определите их, используя только базовый вариант паттерн матчинга как в лекциях.

Сначала введём дополнительную конструкцию:
\begin{align*}
& sub : \mathbb{N} \times \mathbb{N} \to \mathbb{N} \\
& sub (x, 0) = x \\
& sub (x, S(y)) = pred (sub(x, y))
\end{align*}

Очевидно, эта функция является примитивно-рекурсивной. После этого становится возможным ввести и функции $min, max$:

\begin{align*}
& min : \mathbb{N} \times \mathbb{N} \to \mathbb{N} \\
& min(0,y) = 0 \\
& min(S(x), y) = sub(S(x + y), max(S(x), y)) \\
& \\
& max : \mathbb{N} \times \mathbb{N} \to \mathbb{N} \\
& max(0,y) = y \\
& max(S(x), y) = y + sub(S(x), y)
\end{align*}
\item Докажите, что $\forall x \forall y\ (min(x,y) \leq max(x,y))$, где $a \leq b$ означает $\exists c\ (a + c = b)$.

\textbf{Решение.} Докажем утверждение $\varphi(x) =  \forall y\ (min(x,y) \leq max(x,y))$ по индукции:
\begin{itemize}
	\item База. $\varphi(0) = \forall y \ (min(0,y) \leq max(0,y))$. Или, $\forall y \ 0 \leq y$. Заметим, что 
	$\forall y \exists c = y \ \left[ 0 + y = y\right]$.
	\item Переход. $\forall y\ (min(x,y) \leq max(x,y)) \to \forall y\ (min(S(x),y) \leq max(S(x),y))$. 
	Утверждение $\forall y\ (min(S(x),y) \leq max(S(x),y))$ докажем по индукции по $y$ (предположение индукции 
	даёт нам $\forall y\exists c \ min(x,y) + c = max(x,y)$): 
	\begin{itemize}
		\item База $min(S(x),0) \leq max(S(x),0)$. Можем вычислить: $0 \leq S(x)$, значит $\exists \hat{c} = S(x) 
		\ 0 + \hat{c} = S(x)$
		\item Переход. $min(S(x),S(y)) \leq max(S(x),S(y))$, вычислим: $S(min(x,y)) \leq S(max(x,y))$. Это 
		значит: $\exists \tilde{c} : S(min(x, y)) + \tilde{c} = S(max(x, y))$, вычислим левую часть: $\exists 
		\tilde{c} : S(min(x, y) + \tilde{c}) = S(max(x, y))$. Воспользуемся $pred$, получим $\exists \tilde{c} : 
		min(x, y) + \tilde{c} = max(x, y)$. А такое $\tilde{c}$ у нас уже есть: $\tilde{c} = c$.
	\end{itemize}
\end{itemize}
\end{enumerate}

\item Пусть в $PRCPA$ у нас есть функции $+$ и $+'$, определённые следующим образом:
\begin{align*}
& + : \mathbb{N} \times \mathbb{N} \to \mathbb{N} \\
& 0 + y = y \\
& S(x) + y = S(x + y) \\
& \\
& +' : \mathbb{N} \times \mathbb{N} \to \mathbb{N} \\
& x +' 0 = x \\
& x +' S(y) = S(x +' y)
\end{align*}
Докажите, что $\forall x \forall y\ (x + y = x +' y)$

\textbf{Решение.} Докажем утверждение $\varphi(x) = \forall y \ (x + y = x +' y)$ по индукции.
\begin{itemize}
	\item База: $\varphi(0) = \forall y \ 0 + y = 0 +' y$. Вместо этого достаточно показать, что $y = 0 +' y$. 
	Это можем показать по индукции:
	\begin{itemize}
		\item База $0 = 0 +' 0 = 0$ - Верно.
		\item Индукционный переход $y = 0 +' y \to S(y) = 0 +' S(y)$
		\begin{align*}
			& S(y) = 0 +' S(y) \\ 
			& S(y) = S(0 +' y) \\
			& y = 0 +' y
		\end{align*}
		Доказано.
	\end{itemize}
	\item Покажем индукционный переход $S(x) + y = S(x) +' y$. Заметим, что $S(x) + y = S(x + y) = S(x +' y) = x 
	+' S(y)$. Значит будет достаточно показать, что $x +' S(y) = S(x) +' y$. Сделаем это так же по индукции по 
	$y$.
	\begin{itemize}
		\item База $y = 0$. $x +' S(0) = S(x) +' 0$. Это верно по рефлексивности.
		\item Переход. Пусть верно $x +' S(y) = S(x) +' y$, покажем, что верно $x +' S(S(y)) = S(x) +' S(y)$.
		\begin{align*}
			& x +' S(S(y)) = S(x) +' S(y) \\
			& S(x +' S(y)) = S(S(x) +' y) \\
			& S(S(x) +' y) = S(S(x) +' y) 
		\end{align*}
		Доказано.
	\end{itemize}
\end{itemize}

\item Докажите в $CPA$, что $\forall n\ (2^n \leq ack\,n\,n)$, где $a \leq b$ означает $\exists c\ (a + c = b)$ 
и функции $2^{(-)}$ и $ack$ определены следующим образом:
\begin{align*}
& 2^{(-)} : \mathbb{N} \to \mathbb{N} \\
& 2^0 = S(0) \\
& 2^{S(n)} = 2 \cdot 2^n \\
& \\
& ack : \mathbb{N} \to \mathbb{N} \to \mathbb{N} \\
& ack\,0\,n = S(n) \\
& ack\,(S\,m)\,0 = ack\,m\,(S\,0) \\
& ack\,(S\,m)\,(S\,n) = ack\,m\,(ack\,(S\,m)\,n)
\end{align*}

\textbf{Решение.} Заметим, что функция Аккермана монотонна хотя бы по первому аргументу.

Так же, раз мы можем вычислять значения, проверим неравенство для $n = 0, 1, 2$. Оно выполнено:
\begin{itemize}
	\item $2^0 = 1 \le 1 = ack\ 0 \ 0$
	\item $2^1 = 2 \le 3 = ack\ 0 \ 0$
	\item $2^2 = 4 \le 7 = ack\ 0 \ 0$
\end{itemize}

Для остальных $n$ значение $ack \ n \ n$ можно заменить на $ack \ 3 \ n$, которое равно $2^{n + 3} - 3$. 
Очевидно, что $2^n \le 2^{n + 3} - 3$. Так же, в силу монотонности функции аккермана $\forall n \ge 3 \left[ 
ack \ 3 \ n \le ack \ n \ n \right] $. Значит и требуемое неравенство тоже выполнено.

\item Пусть в $CPA$ определена следующая функция:
\begin{align*}
& f : \mathbb{N} \to \mathbb{N} \to \mathbb{N} \\
& f\,0\,x = 2 \cdot x \\
& f\,(S\,n)\,x = f\,n\,(f\,n\,(S\,x))
\end{align*}
Является ли эта функция примитивно рекурсивной? Если да, запишите ее определение (в некарированном виде) в 
$PRCPA$.

\textbf{Решение}
\begin{align*}
& f : \mathbb{N} \times \mathbb{N} \to \mathbb{N} \\
& f \ (0,x) = 2 \cdot x \\
& f \ (n, S(x)) = 2 ^ {2 ^ n} + f \ (n, x) \\
& f \ (S(n), 0) = f \ (n, 0) + 2 ^ {2 ^ {n - 1}} \left[ f \ (n, 0) + 2 ^ {2 ^ {n - 1}}\right]
\end{align*}

\end{enumerate}
