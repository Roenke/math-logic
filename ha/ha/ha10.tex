\section{Расширения арифметики}

\begin{enumerate}

\item Докажите, что $L_1^{lem}$ является консервативным расширением $L_0^{lem}$.

\textbf{Решение.} 

Достаточно привести эквивалентную формулу в классической логике для выражения с конъюнкцией $\varphi \land \psi$. 
Можно убедиться (перебрав значения все значения $\varphi, \psi$ в булевой интерпретации $\{\bot,\top\}$), что 
конъюнкции соответствует формула:
\begin{equation*}
	\neg(\varphi \to \neg \psi)
\end{equation*}

\item Если в $PRCPA$ можно определять функции при помощи паттерн матчинга на нескольких аргументах сразу, то мы 
легко можем определить функции $min$ и $max$:
\begin{align*}
& min : \mathbb{N} \times \mathbb{N} \to \mathbb{N} \\
& min(0,y) = 0 \\
& min(S(x),0) = 0 \\
& min(S(x),S(y)) = S(min(x,y)) \\
& \\
& max : \mathbb{N} \times \mathbb{N} \to \mathbb{N} \\
& max(0,y) = y \\
& max(S(x),S(y)) = S(max(x,y))
\end{align*}
\begin{enumerate}
\item Определите их, используя только базовый вариант паттерн матчинга как в лекциях.

Сначала введём дополнительную конструкцию:
\begin{align*}
& sub : \mathbb{N} \times \mathbb{N} \to \mathbb{N} \\
& sub (x, 0) = x \\
& sub (x, S(y)) = pred (sub(x, y))
\end{align*}

Очевидно, эта функция является примитивно-рекурсивной. После этого становится возможным ввести и функции 
$min, max$:

\begin{align*}
& min : \mathbb{N} \times \mathbb{N} \to \mathbb{N} \\
& min(0,y) = 0 \\
& min(S(x), y) = sub(S(x + y), max(S(x), y)) \\
& \\
& max : \mathbb{N} \times \mathbb{N} \to \mathbb{N} \\
& max(0,y) = y \\
& max(S(x), y) = y + sub(S(x), y)
\end{align*}
\item Докажите, что $\forall x \forall y\ (min(x,y) \leq max(x,y))$, где $a \leq b$ означает $\exists c\ (a + 
c = b)$.

\textbf{Решение.} Докажем утверждение $\varphi(x) =  \forall y\ (min(x,y) \leq max(x,y))$ по индукции:
\begin{itemize}
	\item База. $\varphi(0) = \forall y \ (min(0,y) \leq max(0,y))$. Или, $\forall y \ 0 \leq y$. Заметим, что 
	$\forall y \exists c = y \ \left[ 0 + y = y\right]$.
	\item Переход. $\forall y\ (min(x,y) \leq max(x,y)) \to \forall y\ (min(S(x),y) \leq max(S(x),y))$. 
	Утверждение $\forall y\ (min(S(x),y) \leq max(S(x),y))$ докажем по индукции по $y$ (предположение индукции 
	даёт нам $\forall y\exists c \ min(x,y) + c = max(x,y)$): 
	\begin{itemize}
		\item База $min(S(x),0) \leq max(S(x),0)$. Можем вычислить: $0 \leq S(x)$, значит $\exists \hat{c} = S(x) 
		\ 0 + \hat{c} = S(x)$
		\item Переход. $min(S(x),S(y)) \leq max(S(x),S(y))$, вычислим: $S(min(x,y)) \leq S(max(x,y))$. Это 
		значит: $\exists \tilde{c} : S(min(x, y)) + \tilde{c} = S(max(x, y))$, вычислим левую часть: $\exists 
		\tilde{c} : S(min(x, y) + \tilde{c}) = S(max(x, y))$. Воспользуемся $pred$, получим $\exists \tilde{c} : 
		min(x, y) + \tilde{c} = max(x, y)$. А такое $\tilde{c}$ у нас уже есть: $\tilde{c} = c$.
	\end{itemize}
\end{itemize}
\end{enumerate}

\item Пусть в $PRCPA$ у нас есть функции $+$ и $+'$, определённые следующим образом:
\begin{align*}
& + : \mathbb{N} \times \mathbb{N} \to \mathbb{N} \\
& 0 + y = y \\
& S(x) + y = S(x + y) \\
& \\
& +' : \mathbb{N} \times \mathbb{N} \to \mathbb{N} \\
& x +' 0 = x \\
& x +' S(y) = S(x +' y)
\end{align*}
Докажите, что $\forall x \forall y\ (x + y = x +' y)$

\textbf{Решение.} Докажем утверждение $\varphi(x) = \forall y \ (x + y = x +' y)$ по индукции.
\begin{itemize}
	\item База: $\varphi(0) = \forall y \ 0 + y = 0 +' y$. Вместо этого достаточно показать, что $y = 0 +' y$. 
	Это можем показать по индукции:
	\begin{itemize}
		\item База $0 = 0 +' 0 = 0$ - Верно.
		\item Индукционный переход $y = 0 +' y \to S(y) = 0 +' S(y)$
		\begin{align*}
			& S(y) = 0 +' S(y) \\ 
			& S(y) = S(0 +' y) \\
			& y = 0 +' y
		\end{align*}
		Доказано.
	\end{itemize}
	\item Покажем индукционный переход $S(x) + y = S(x) +' y$. Заметим, что $S(x) + y = S(x + y) = S(x +' y) = x 
	+' S(y)$. Значит будет достаточно показать, что $x +' S(y) = S(x) +' y$. Сделаем это так же по индукции по 
	$y$.
	\begin{itemize}
		\item База $y = 0$. $x +' S(0) = S(x) +' 0$. Это верно по рефлексивности.
		\item Переход. Пусть верно $x +' S(y) = S(x) +' y$, покажем, что верно $x +' S(S(y)) = S(x) +' S(y)$.
		\begin{align*}
			& x +' S(S(y)) = S(x) +' S(y) \\
			& S(x +' S(y)) = S(S(x) +' y) \\
			& S(S(x) +' y) = S(S(x) +' y) 
		\end{align*}
		Доказано.
	\end{itemize}
	Доказано.
\end{itemize}

\item Докажите в $CPA$, что $\forall n\ (2^n \leq ack\,n\,n)$, где $a \leq b$ означает $\exists c\ (a + c = b)$ 
и функции $2^{(-)}$ и $ack$ определены следующим образом:
\begin{align*}
& 2^{(-)} : \mathbb{N} \to \mathbb{N} \\
& 2^0 = S(0) \\
& 2^{S(n)} = 2 \cdot 2^n \\
& \\
& ack : \mathbb{N} \to \mathbb{N} \to \mathbb{N} \\
& ack\,0\,n = S(n) \\
& ack\,(S\,m)\,0 = ack\,m\,(S\,0) \\
& ack\,(S\,m)\,(S\,n) = ack\,m\,(ack\,(S\,m)\,n)
\end{align*}

\textbf{Решение.}

Докажем вспомогательное утверждение: $\forall n \forall m \ ack \ n \ m \geq S(m)$. Индукция по $m$.
\begin{itemize}
	\item База. $m = 0: \ \forall n \ ack \ n \ S(0) \geq 0$. Это всегда правда.
	\item Переход. $\forall n \ ack \ n \ m \geq S(m) \to \forall n \ ack \ n \ S(m) \geq S(S(m))$.
	
	А теперь воспользуемся индукцией по $n$. 
	\begin{itemize}
		\item База $n = 0:$ $ack \ 0 \ S(m) = S(S(m)) \geq S(S(m))$. Верно.
		\item Переход $ack \ n \ S(m) \geq S(S(m)) \to \ ack \ S(n) \ S(m) \geq S(S(m))$.
		\begin{align*}
		& ack \ S(n) \ S(m) \geq S(S(m)) \\
		& ack \ n \ (ack \ S(n) \ m) \geq S(ack \ S(n) \ m) \geq S(S(m)) 
		\end{align*}
		Доказано
	\end{itemize}
	Доказано
\end{itemize}

Теперь покажем, что функция Аккермана монотонна по обоим аргументам.
\begin{itemize}
	\item По второму:
	\begin{align*}
		ack \ S(n) \ S(m) = ack \ n \ (ack S(n) \ m) \geq ack \ S(n) \ m 
	\end{align*}
	\item По первому (Пользуясь тем, что монотонна по второму и ранее доказанным утверждением):
	\begin{align*}
		ack \ S(n) \ S(m) = ack \ n \ (ack S(n) \ m) \geq ack \ n \ S(m)
	\end{align*}
\end{itemize}

Теперь можем зафиксировать первый параметр функции Аккермана, и показать, что требуемое неравенство выполнено. 
Заметим, что $ack \ 0 \ n = n + 1$, Теперь, покажем, что $ack \ 1 \ n = n + 2$ (по индукции) 
\begin{itemize}
	\item База $n = 0$: $ack \ 1 \ 0 = ack \ 0 \ 1 = S(1) = 2$ - верно.
	\item Переход: $ack \ 1 \ n = n + 2 \ to ack \ 1 \ (n + 1) = n + 3$
	\begin{align*}
		ack \ 1 \ (n + 1) = ack \ 0 (ack \ 1 \ n) = n + 2 + 1 = n + 3
	\end{align*}
	Верно.
\end{itemize}

Теперь, докажем, что $ack \ 2 \ n = 2 n + 3$ (Снова индукция)
\begin{itemize}
	\item База $n = 0 : ack \ 2 \ 0 = ack \ 1 \ 1 = 1 + 2 = 3$ - верно.
	\item Переход: $ack \ 2 \ n = 2 n + 3 \to ack \ 2 \ (n + 1) = 2 n + 5$
	\begin{align*}
		ack \ 2 \ (n + 1) = ack \ 1 \ (ack \ 2 \ n) = ack \ 2 \ n + 2 = 2 n + 3 + 2 = 2 n + 5
	\end{align*}
	Верно.
\end{itemize}

Покажем, что $ack \ 3 \ n = 2 ^ {n + 3} - 3$. (и снова индукция)
\begin{itemize}
	\item База $n = 0 : ack \ 3 \ 0  = ack \ 2 \ 1 = 2 * 1 + 3 = 5 = 2^3 - 3$ - верно
	\item Переход: $ack \ 3 \ n = 2 ^ {n + 3} - 3 \to ack \ 3 \ (n + 1) = 2 ^ {n + 4} - 3$
	\begin{align*}
		ack \ 3 \ (n + 1) = ack \ 2 \ (ack \ 3 \ n) = 2 (2 ^ {n + 3} - 3) + 3 = 2 ^ {n + 4} - 6 + 3 = 
		2 ^ {n + 4} - 3
	\end{align*}
	Верно
\end{itemize}

Проверим неравенство для $n = 0, 1, 2$. Оно выполнено:
\begin{itemize}
	\item $2^0 = 1 \le 1 = ack\ 0 \ 0$
	\item $2^1 = 2 \le 3 = ack\ 1 \ 1$
	\item $2^2 = 4 \le 7 = ack\ 2 \ 2$
\end{itemize}

Для остальных $n$ значение $ack \ n \ n$ можно заменить на $ack \ 3 \ n$ (в силу монотонности). 
Очевидно, что $\forall n \leq 3 \ \left[2^n \le 2^{n + 3} - 3\right]$. 

\item Пусть в $CPA$ определена следующая функция:
\begin{align*}
& f : \mathbb{N} \to \mathbb{N} \to \mathbb{N} \\
& f\,0\,x = 2 \cdot x \\
& f\,(S\,n)\,x = f\,n\,(f\,n\,(S\,x))
\end{align*}
Является ли эта функция примитивно рекурсивной? Если да, запишите ее определение (в некарированном виде) в 
$PRCPA$.

\textbf{Решение}
\begin{align*}
& f : \mathbb{N} \times \mathbb{N} \to \mathbb{N} \\
& f \ (0,x) = 2 \cdot x \\
& f \ (n, S(x)) = 2 ^ {2 ^ n} + f \ (n, x) \\
& f \ (S(n), 0) = f \ (n, 0) + 2 ^ {2 ^ n} \left[ f \ (n, 0) + 2 ^ {2 ^ n}\right]
\end{align*}

Пояснение: Первое выражение переписали без изменений. Заметим, что при фиксированном первом аргументе $n$ по 
второму аргументу функция растёт равномерно с шагом $2^{2 ^ n}$ (так получается второе выражение). Теперь хотим 
упростить выражение $f \ (S(n), 0) = f \ (n, f \ (n, S(0)))$ Значит, чтобы вычислить значение $f(S(n), 0)$ нужно 
взять $f(n, 0)$ и добавить к нему $2^{2^n}$, ровно $f(n, S(0))$ раз. Но мы знаем, что $f(n, S(0))$ можно записать 
так: $f(n, 0) + 2^{2^n}$. Так получается третье выражение.

\end{enumerate}
