\section*{Логика первого порядка}

\begin{enumerate}

\item Определите формулы, удовлетворяющие следующим описаниям.
    Для первых двух заданий мы предполагаем, что в сигнатуре есть предикатный символ $\leq$.
\begin{enumerate}
\item $\exists x \leq t\ \psi$ (``Существует $x$, меньше либо равный $t$, такой что верно $\psi$'').
\begin{equation*}
	\exists x \ x \leq t \land \psi(x)
\end{equation*}
\item $\forall x \leq t\ \psi$ (``Для любого $x$, меньше либо равного $t$, верно $\psi$'').
\begin{equation*}
\forall x \ x \leq t \to \psi(x)
\end{equation*}
\item ``Существует не менее двух элементов, удовлетворяющих $\varphi(x)$''.
\begin{equation*}
\exists x \exists y \ (x \neq y) \land \varphi(x) \land \varphi(y)
\end{equation*}
\item ``Существует ровно два элемента, удовлетворяющие $\varphi(x)$''.
\begin{equation*}
	\exists x \exists y \ (x \neq y) \land \varphi(x) \land \varphi(y) \land (\forall z \ \varphi(z) \to (z = x \lor z = y))
\end{equation*}
\item ``Существует по крайней мере один, но не более двух элементов, удовлетворяющих $\varphi(x)$''.
\begin{equation*}
	\exists x \ \varphi(x) \land (\forall p \forall y \forall z \ \varphi(p) \land \varphi(y) \land \varphi(z) \to ((p = y) \lor (y = z) \lor (p = z)))
\end{equation*}

\item ``Существует не более одного элемента, удовлетворяющего $\varphi(x)$''
\begin{equation*}
	\forall x \forall y \ (\varphi(x) \land \varphi(y) \to (x = y)
\end{equation*}
\end{enumerate}

\item Напишите на хаскелле функцию, аналогичную конструкции $\mathbf{case}$ для пар, используя $fst$ и $snd$.
    Укажите ее тип (вам нужно будет использовать функции высшего порядка вместо расширения контекста).
    Реализуйте функции $fst'$ и $snd'$, эквивалентные обычным $fst$ и $snd$, через эту функцию.

\textbf{Решение.} Если я правильно понял, что требуется, то может определись функцию $fun$, и $fst', snd'$ через неё:

\lstinputlisting[language=haskell]{ha/code/task2.hs}

\item Пусть у нас есть несколько формул:
\begin{enumerate}
\item \label{it:no} $x \neq y$ -- $\{M_0\}$
\item \label{it:e} $\exists x (x \neq y)$ -- $\{M_2\}$
\item \label{it:a} $\forall x (x \neq y)$ -- $\{M_0\}$
\item \label{it:ee} $\exists x \exists y (x \neq y)$ -- $\{M_2\}$
\item \label{it:ea} $\exists x \forall y (x \neq y)$ -- $\{\}$
\item \label{it:ae} $\forall x \exists y (x \neq y)$ -- $\{M_2\}$
\item \label{it:aa} $\forall x \forall y (x \neq y)$ -- $\{M_0\}$
\end{enumerate}
И несколько интерпретаций:
\begin{align*}
M_0 & = \varnothing \\
M_1 & = \{ 7 \} \\
M_2 & = \{ 13, 28 \}
\end{align*}
Какие из этих формул верны в каких моделях?

\item Докажите, что формулы $\forall x \forall y (x \neq y)$ и $\neg \exists x\ \top$ эквивалентны,
    написав лямбда терм типа $((\forall x \forall y (x \neq y)) \to \neg \exists x\ \top) \land ((\neg \exists x\ 
    \top) \to \forall x \forall y (x \neq y))$.

\item Пусть теория $T$ состоит из одной аксиомы $\{\,st : \exists (x : s)\ \top\,\}$.
    Пусть $x \notin FV(\varphi)$.
    Тогда докажите, что следующие формулы являются теоремами этой теории.
    Приведите и дерево вывода, и лямбда терм, доказывающие эти формулы.
\begin{enumerate}
\item $\varphi \land (\forall x\ \psi) \to \forall x\ (\varphi \land \psi)$ (эта формула выводима и в пустой 
теории)
\begin{equation*}
	\lambda p. \mathbf{case} \ p \ of \{ (phi, f) \to \lambda x.(phi, f \ x) \}
\end{equation*}
\item $(\forall x\ (\varphi \land \psi)) \to \varphi \land (\forall x\ \psi)$
\begin{equation*}
	\lambda f. \mathbf{case} \ st \ of \{ (phi, psi) \to (f\ phi, \lambda x. snd \ (f \ x)) \}
\end{equation*}
\end{enumerate}

\item Докажите, что следующие формулы выводимы (в пустой теории), написав лямбда термы, доказывающие их.
    Исключенное третье можно (и нужно) использовать только в последнем пункте.
\begin{enumerate}
\item $(\forall x\ (\neg \varphi)) \to \neg \exists x\ \varphi$

Или, что то же самое: $\forall x\ (\varphi \to \bot) \to \exists x\ \varphi \to \bot$
\begin{equation*}
	\lambda f.\lambda p. \mathbf{case} \ p \ of \ \{(x, pr) \to (f \ x) \ pr\}
\end{equation*}
\item $(\neg \exists x\ \varphi) \to \forall x\ (\neg \varphi)$

Или, что то же самое:  $((\exists x\ \varphi) \to \bot) \to \forall x\ (\varphi \to \bot)$
\begin{equation*}
	\lambda f.\lambda x. \lambda \varphi .f \ (x, \varphi) 
\end{equation*}
\item $\exists x\ (\neg \varphi) \to \neg \forall x\ \varphi$

Или, что то же самое:  $\exists x\ (\varphi \to \bot) \to \forall x\ \varphi \to \bot$
\begin{equation*}
	\lambda p. \lambda f. \mathbf{case} \ p \ of \ \{(x, pr) \to pr \ (f \ x)\}
\end{equation*}
\item $(\neg \forall x\ \varphi) \to \exists x\ (\neg \varphi)$

Или, что то же самое:  $((\forall x\ \varphi) \to \bot) \to \exists x\ (\varphi \to \bot)$. Обозначим это 
выражение как $\psi$. Предположим, что у нас есть исключённое третье. Тогда достаточно доказать, что верно $\neg 
\neg \psi = ((((\forall x\ \varphi) \to \bot) \to \exists x\ (\varphi \to \bot)) \to \bot) \to \bot$.
\begin{equation*}
	\lambda f. f (\lambda f_1.\lambda p. \mathbf{case} \ p \ of \{ (x, pr) \to f_1 (\lambda x'.x') \})
\end{equation*}
\end{enumerate}

\end{enumerate}

