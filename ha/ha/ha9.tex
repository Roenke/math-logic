\section*{Формальная арифметика}

В заданиях, где требуется привести доказательство, нужно привести словесное описание доказательства.
Необходимо явно указать, где и какие аксиомы арифметики Пеано были использованы.
Все доказательства должны быть в интуиционистской логике.
\begin{enumerate}
	
\item Определите формулу $\varphi(x,y)$, задающую график функции $pred$, удовлетворяющей следующим условиям:
\begin{align*}
pred(0) & = 0 \\
pred(S(x)) & = x
\end{align*}
Докажите, что $\forall x \exists! y (\varphi(x,y))$

\textbf{Решение.}
\begin{align*}
	\varphi(0, y) &= 0 = y \\
	\varphi(S(x), y) &= x = y
\end{align*}

Утверждение $\forall x \exists! y (\varphi(x,y))$ следует из единственности элементов:
\begin{align*}
	\forall &x \ 0 \neq S(x)\\
	\forall &x \forall y \ S(x) = S(y) \to x = y
\end{align*}

\item Докажите, что аксиомы для сложения определяют его уникальным образом.
    То есть если мы добавим в сигнатуру новый функциональный символ $+'$ и новые аксиомы
\begin{align*}
\forall y\ & 0 +' y = y \tag{$+'0$} \\
\forall x \forall y\ & S(x) +' y = S(x +' y) \tag{$+'S$},
\end{align*}
то в ней будет доказуема формула $\forall x \forall y\ (x + y = x +' y)$.

\item Добавим в сигнатуру функциональные символы $exp$ и $fac$ для функций возведения в степень и факториала
соответственно.
    Напишите аксиомы для этих функциональных символов, определяющие их уникальным образом.

	\textbf{Решение.}
	\begin{align*}
		x \ exp \ 0 &= S(0)\\
		x \ exp \ S(y) &= x \cdot (x \ exp\ y) \\ 
		fac \ 0 &= S(0) \\
		fac \ S(x) &= S(x) \cdot fac\ x
	\end{align*}
	

\item Докажите следующие свойства:
\begin{enumerate}
\item $\forall x \forall y\ (x + y = 0 \to x = 0 \land y = 0)$.
\item $\forall x \forall y\ (x \cdot y = 0 \to x = 0 \lor y = 0)$.
\end{enumerate}

\item Докажите коммутативность сложения.

\end{enumerate}

