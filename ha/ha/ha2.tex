

Мы будем говорить, что два множества разрешимо равномощны, если у этих множеств есть аналоги в виде типов языка хаскелл,
    и между этими типами существуют взаимно обратные функции языка хаскелл
Аналогом множества $\mathbb{N}$ является тип $data\ Nat\ = Zero\ |\ Suc\ Nat$.
Аналогом множества $\{0,1\}^*$ является тип $[Bool]$.
Если $X$ -- подмножество $\{0,1\}^*$, то его аналогом является тип $type X = [Bool]$, но предполагается,
    что в функции $X \to a$ не передаются аргументы, выходящие за пределы $X$, и функции $a \to X$ не возвращают результат, выходящий за его пределы.

\begin{enumerate}

\item Докажите, что $\mathbb{N}$ и $\{0,1\}^*$ разрешимо равномощны, где второе множество -- это множество последовательностей из 0 и 1.

\item Докажите, что $\{0,1\}^*$ и $\mathbb{N}_2$ разрешимо равномощны, где второе множество -- это множество двоичных натуральных чисел,
    то есть последовательностей 0 и 1 без ведущих нулей (кроме случая, когда последовательность состоит из одной цифры).

\item Докажите, что $\{0,1\}^*$ и множество корректных программ на каком-либо (любом) языке программирования разрешимо равномощны.

\item Определите множество простых чисел.

\item Определите следующие функции над $\mathbb{Q}$ и докажите их корректность:
\begin{enumerate}
\item Функция $neg : \mathbb{Q} \to \mathbb{Q}$, возвращающая обратное по сложению число.
\item Функция $inv : \mathbb{Q}_{\neq 0} \to \mathbb{Q}_{\neq 0}$, возвращающая обратное по умножению число.
\item Функция $plus : \mathbb{Q} \times \mathbb{Q} \to \mathbb{Q}$, возвращающая сумму двух чисел.
\end{enumerate}

\item Докажите, что существует биекция между двумя вариантами определения множества $\Pi (a \in A) B_a$, приведенных в лекции.

\item \label{it:vec}
    Пусть $Vec(A,n)$ -- множество списков длины $n$, элементы которых лежат в множестве $A$.
    В лекции был приведен пример функции $index$.
    Опишите аналогичным образом ``тип'' функций (то есть в каком множестве они лежат, все эти множества будут множествами зависимых функций), приведенных ниже.
    Каждая из этих функций должна принимать и возвращать элементы множеств вида $Vec(A,n)$ и, возможно, другие аргументы.
\begin{enumerate}
\item Функция $reverse$, разворачивающая список.
\item Функция $append$, конкатенирующая два списка.
\item Функция $filter$, принимающая предикат и список длины $n$, и возвращающая
\end{enumerate}

\item Задания на хаскелле:
\begin{enumerate}
\item См. cb.hs.
\item Пусть $\mathbb{N}_{\geq 2} = \{ n \in \mathbb{N}\ |\ n \geq 2 \}$ и $m : \mathbb{N}_{\geq 2} \times \mathbb{N}_{\geq 2} \to \mathbb{N}$ вовзращает произведение чисел, то есть $m(x,y) = x \cdot y$.
    На лекции мы видели, что существует отношение эквивалентности $\sim$ на $\mathbb{N}_{\geq 2} \times \mathbb{N}_{\geq 2}$, такое что $(\mathbb{N}_{\geq 2} \times \mathbb{N}_{\geq 2})/\!\!\sim$ равномощно $im(m)$.
    Задайте тип на хаскелле, аналогичный $(\mathbb{N}_{\geq 2} \times \mathbb{N}_{\geq 2})/\!\!\sim$ (вам понадобится задать $instance\ Eq$ для него).
    Определите биекцию на хаскелле между этим типом и $im(m)$.
\item
{
    Определите на хаскелле два варианта рациональных чисел: один через отношение эквивалентности, другой через канонические представители.
}
    Определите биекцию между ними.
\end{enumerate}

\item Опциональная задача для любителей программирования с зависимыми типами. Реализуйте функции из задания \ref{it:vec} на агде.

\end{enumerate}

\end{document}
