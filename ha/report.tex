\documentclass[fleqn]{article}
\usepackage{cmap}
\usepackage[left=1in, right=1in, top=1in, bottom=1in]{geometry}
\usepackage{mathexam}
\usepackage{mathtext} 				% русские буквы в фомулах
\usepackage[T2A]{fontenc}			% кодировка
\usepackage[utf8]{inputenc}			% кодировка исходного текста
\usepackage[english,russian]{babel}	% локализация и переносы
\usepackage{enumerate}
%%% Дополнительная работа с математикой
\usepackage{amsmath,amsfonts,amssymb,amsthm,mathtools,amsthm} % AMS
\usepackage{icomma} % "Умная" запятая: $0,2$ --- число, $0, 2$ --- перечисление
\usepackage{graphicx}

% Двойные квадратные скобки
\usepackage{stmaryrd}
\newcommand{\llb}{\llbracket}
\newcommand{\rrb}{\rrbracket}

%% Шрифты
\usepackage{euscript}	 % Шрифт Евклид
\usepackage{mathrsfs} % Красивый матшрифт
\ExamClass{SE Academic University}
\ExamName{Homework 2}
\ExamHead{\today}

%% Шрифты
\usepackage{euscript}	 % Шрифт Евклид
\usepackage{mathrsfs} % Красивый матшрифт

%Листинг кода
%\usepackage{listings}
\usepackage{listingsutf8}
\usepackage{color}
\usepackage{bussproofs}
\renewcommand{\qedsymbol}{$\blacksquare$}
%Для листинга кода
\definecolor{mygreen}{rgb}{0,0.6,0}
\definecolor{mygray}{rgb}{0.5,0.5,0.5}
\definecolor{mymauve}{rgb}{0.63,0.082,0.082}


\lstset{
	inputencoding=utf8,
	%
	backgroundcolor=\color{white},   % choose the background color; you must add \usepackage{color} or \usepackage{xcolor}
	basicstyle=\footnotesize,        % the size of the fonts that are used for the code
	breakatwhitespace=false,         % sets if automatic breaks should only happen at whitespace
	breaklines=true,                 % sets automatic line breaking
	captionpos=b,                    % sets the caption-position to bottom
	commentstyle=\color{black},    % comment style
	deletekeywords={...},            % if you want to delete keywords from the given language
	escapeinside={\%*}{*)},          % if you want to add LaTeX within your code
	extendedchars=\true,              % lets you use non-ASCII characters; for 8-bits encodings only, does not work with UTF-8
	frame=false,                    % adds a frame around the code
	keepspaces=true,                 % keeps spaces in text, useful for keeping indentation of code (possibly needs columns=flexible)
	keywordstyle=\color{blue},       % keyword style
	morekeywords={*,...},            % if you want to add more keywords to the set
	numbers=left,                    % where to put the line-numbers; possible values are (none, left, right)
	numbersep=5pt,                   % how far the line-numbers are from the code
	numberstyle=\tiny\color{black}, % the style that is used for the line-numbers
	rulecolor=\color{white},         % if not set, the frame-color may be changed on line-breaks within not-black text (e.g. comments (green here))
	showspaces=false,                % show spaces everywhere adding particular underscores; it overrides 'showstringspaces'
	showstringspaces=false,          % underline spaces within strings only
	showtabs=false,                  % show tabs within strings adding particular underscores
	stepnumber=1,                    % the step between two line-numbers. If it's 1, each line will be numbered
	stringstyle=\color{black},     % string literal style
	tabsize=4                  % sets default tabsize to 2 spaces
	% show the filename of files included with \lstinputlisting; also try caption instead of title
}   


\let\ds\displaystyle

\begin{document}

%\section{Множества}
\begin{enumerate}

\item Определите множество частичных функций через множество их графиков.

\textbf{Решение.} 

Множество подмножеств $G \subseteq A \times (B \cup \{undefined\})$,таких что для любого $a \in A$ существует 
единственный $b \in (B \cup \{undefined\})$, такой что $(a, b) \in G$.

\item Композиция функций $f : A \to B$ и $g : B \to C$ -- это функция $g \circ f : A \to C$, такая что $(g 
\circ f)(a) = g(f(a))$.
    В хаскелле есть аналог этой операции -- это $g\ .\ f$. Задайте график функции $g \circ f$ через графики 
    $f$ и $g$.

\textbf{Решение.} 

Рассмотрим графики функций $f$ и $g$. Обозначим $G_f \subseteq A \times B$, $G_g \subseteq B \times C$. 
Обозначим $B_f$ множество элементов $b \in B \exists a \in A : (a, b) \in G_f$. А в качестве $B_g$ обозначим 
множество элементов $b\in B : \exists c \in C : (b, c) \in G_g$. Тогда введём $B = B_f \cap B_g$. Тогда в 
качестве графика функции $g \circ f$ можно взять множества пар $(a, c)$ таких, что $\exists b \in B : (a, b) 
\in G_f \land (b, c) \in G_g$.  

\item Докажите, что если $A$ вкладывается в $B$ и $B$ вкладывается в $C$, то $A$ вкладывается в $C$.

\textbf{Решение.} 

$A$ вкладывается в $B$ $\Rightarrow \exists f:A\rightarrow B - $ инъекция.

$B$ вкладывается в $C$ $\Rightarrow \exists g:B\rightarrow C - $ инъекция.

Покажем, что композиция $g \circ f : A\rightarrow C - $ является инъекцией.

Для этого достаточно показать, что если $(g \circ f)(a) = (g \circ f)(a')$ то $a = a'$. Заметим, что 
$g:B\rightarrow C$ - инъекция, это значит, что $f(a) = f(a')$. $f:A\rightarrow C$ - инъекция, значит $a = a'$.

\item Докажите, что если $A$ накрывает $B$ и $B$ накрывает $C$, то $A$ накрывает $C$.

\textbf{Решение.} 

$A$ накрывает в $B$ $\Rightarrow \exists f:A\rightarrow B - $ сюръекция.

$B$ накрывает в $C$ $\Rightarrow \exists g:B\rightarrow C - $ сюръекция.

Покажем, что композиция $g \circ f : A\rightarrow C - $ является сюръекцией.

Для этого достаточно показать, что $\forall c \in C \exists a \in A : (g \circ f)(a) = c$. заметим, что $g$ - 
сюръекция, значит, $\forall c \exists b \in B : g(b) = c$. Теперь, заметим, что $f$ - сюръекция, значит, 
$\forall b \in B \exists a \in A : f(a) = b$. Объединяя эти два утверждения, получим определение сюръекции 
для $(g \circ f)$. $\forall c \in C \exists b \in B : g(b) = c \exists a \in A : f(a) = b \Rightarrow \forall 
c \in C \exists a \in A : g(f(a)) = c$. Значит композиция сюръекций снова сюръекция.

\item Докажите, что следующие свойства верны:
\begin{enumerate}
\item $A$ равномощно $A$.

\textbf{Решение.} 

Существует тождественное отображение $id: A\rightarrow A$, $id(a) = a$. Т.к. это биекция(для $a\neq \hat{a}$, 
они переходят $id(a) = a \neq \hat{a} = id(\hat{a}),$ и $\forall a\in A \exists\hat{a} = a \in A : 
id(\hat{a}) = a$ ), то $A$ равномощно $A$. 

\item Если $A$ равномощно $B$, то $B$ равномощно $A$.

\textbf{Решение.} 

$A$ равномощно $B$ $\Rightarrow$ $\exists \varphi(a):A\rightarrow B$ - биекция $\Leftrightarrow$ $\exists 
\varphi^{-1} : B \rightarrow A$ - обратная функция - биекция $\Rightarrow$ $B$ равномощно $A$ 

\item Если $A$ равномощно $B$ и $B$ равномощно $C$, то $A$ равномощно $C$.

\textbf{Решение.} 

Из такой постановки задачи следуют постановки задач пунктах $3, 4$. В них мы нашли функцию $\varphi = g \circ 
f$, которая является и накрытием и вложением $A$ в $C$, а значит, инъекцией и сюръекцией $\Rightarrow$ 
$\varphi$ является биекцией $\Rightarrow$ $A$ равномощно $C$.

\end{enumerate}

\item Докажите, что множества $A \amalg B \to C$ и $(A \to C) \times (B \to C)$ равномощны.

\textbf{Решение.} 

Приведем биекцию $\varphi : (A \amalg B \to C) \rightarrow ((A \to C) \times (B \to C))$

\begin{equation*}
	\varphi(f) = (a \mapsto f(Left(a)), b \mapsto f(Right(b)))
\end{equation*}

и обратная к ней

\begin{align*}
	& \varphi^{-1}(g) = Left(x) \mapsto \pi_1(g)(x) \\
	& \varphi^{-1}(g) = Right(x) \mapsto \pi_2(g)(x)
\end{align*}

Несложно убедиться, что эти функции взаимно обратные.

Выберем произвольную функцию $f\in A \amalg B \to C$:
\begin{align*}
	& f(Left(x)) = e \\
	& f(Right(x)) = d
\end{align*}

и вычислим значение $\varphi^{-1}(\varphi(f))$:

\begin{align*}
	& \varphi^{-1}(\varphi(f)) = \pi_1(a \mapsto f(Left(a)), b \mapsto 	f(Right(b))) = f(Left(x)) \\
	& \varphi^{-1}(\varphi(f)) = \pi_2(a \mapsto f(Left(a)), b \mapsto 	f(Right(b))) = f(Right(x))
\end{align*}

Так же несложно показать и обратное преобразование:

\begin{equation*}
\begin{array} {lcl}
	\varphi(\varphi^{-1}(g)) = (a \mapsto \varphi^{-1}(g)(Left(a)), b 	\mapsto \varphi^{-1}(g)(Right(b))) = 
	\\
	= (a \mapsto \pi_1(g)(Left(a)), b \mapsto \pi_2(g)(Right(b))) = (\pi_1(g), \pi_2(g)) = g
\end{array}
\end{equation*}

Таким образом, $\varphi$ - биекция, и множества $A \amalg B \to C$ и $(A \to C) \times (B \to C)$ равномощны.

\item Докажите, что множества $A \times B \to C$ и $A \to (B \to C)$ равномощны.

\textbf{Решение.} 

По аналогии с предыдущим заданием придумаем биекцию $\varphi : (A \times B \to C)\to (A \to (B \to C))$.

\begin{equation*}
	\varphi(f) = x \mapsto (y \mapsto f(x, y))
\end{equation*}

Можно заметить, что получили каррирование функции $f \in A \times B \to C$.

Обратная функция к $\varphi(f)$:

\begin{equation*}
	\varphi^{-1}(g) = p \mapsto g(\pi_1(p))(\pi_2(p))
\end{equation*}

Таким образом, $\varphi$ - биекция, и множества $A \times B \to C$ и $A \to (B \to C)$ равномощны.

\item Докажите, что $|\mathcal{P}(A)| = 2^{|A|}$ и $|A \to B| = |B|^{|A|}$.

\textbf{Решение.} 

Очевидно, первое равенство является прямым следствием второго, т.к. $\mathcal{P}(A)$ имеет 
взаимно-однозначное соответствие с отображениями $A \to \Omega$, где $\Omega = \{ \top, \bot \}$, значит 
$|\mathcal{P}(A)| = |A \to \Omega| = |\Omega|^{|A|} = 2^{|A|}$.

Докажем второе равенство.

Для этого достаточно найти биекцию $A \to B$ на множество 

$\{0, 1, ..., |B|^{|A|} - 1 \}$.

Раз уж $A$ и $B$ конечные, то из элементы можно пронумеровать номерами $\{0, 1, 2,..., |A|\}$ и $\{0, 1, 
2,..., |B|\}$ соответственно. Теперь, если записать последовательно $A$ образов элементов множества $A$ для 
которой $f \in A \to B$, то получим последовательность $f(a_{|A|}), f(a_{|A| - 1}), f(a_{|A| - 2}), .., 
f(a_2), f(a_1)$ элементов из $B$. Их можно отобразить на множество $\{0, 1, 2,..., |B|\}$ Получим число в 
системе счисления с основанием равным $|B|$ и $|A|$ разрядами. Переведя это число по соответствующим правилам 
в десятичную систему исчиления получим значение из интервала $[0; |B|^{|A|} - 1]$. Так же есть возможность 
совершить обратное преобразование (последовательное взятие остатка по модулю $|B|$ и целочисленное деление на 
значение $|B|$). Таким образом, требуемая биекция найдена, следовательно $|A \to B| = |B|^{|A|}$.

\end{enumerate}

%\section{Конструкции над множествами}

\def\doubleunderline#1{\underline{\underline{#1}}}

Мы будем говорить, что два множества разрешимо равномощны, если у этих множеств есть аналоги в виде типов 
языка хаскелл, и между этими типами существуют взаимно обратные функции языка хаскелл. Аналогом множества 
$\mathbb{N}$ является тип $data\ Nat\ = Zero\ |\ Suc\ Nat$. Аналогом множества $\{0,1\}^*$ является тип 
$[Bool]$. Если $X$ -- подмножество $\{0,1\}^*$, то его аналогом является тип $type X = [Bool]$, но 
предполагается, что в функции $X \to a$ не передаются аргументы, выходящие за пределы $X$, и функции $a \to 
X$ не возвращают результат, выходящий за его пределы.

\begin{enumerate}

\item Докажите, что $\mathbb{N}$ и $\{0,1\}^*$ разрешимо равномощны, где второе множество -- это множество 
последовательностей из 0 и 1.

\textbf{Решение.} Построим две инъекции. Применим теорему Кантора-Берштейна. 

$f:\mathbb{N} \rightarrow \{0, 1\}*$. - представление числа в двоичной системе исчисления. Инъективность 
очевидна.

$g:\{0,1\}^* \rightarrow \mathbb{N}$.- добавим к каждому члену последовательности из $0$ и $1$ по единице, 
получим число в троичной системе исчисления без ведущих нулей. Переведем его в десятичную систему исчисление. 
Инъективность следует из отсутствия ведущих нулей и биекцией между троичной и десятичной системами 
исчисления. 

Равномощность следует из теоремы Кантора-Бернштейна.

\item Докажите, что $\{0,1\}^*$ и $\mathbb{N}_2$ разрешимо равномощны, где второе множество -- это множество 
двоичных натуральных чисел, то есть последовательностей 0 и 1 без ведущих нулей (кроме случая, когда     
последовательность состоит из одной цифры).
    
\textbf{Решение.} Заметим, что между $\mathbb{N}_2$ и $\mathbb{N}$ существует биекция - перевод числа в 
двоичную систему и обратно, то задача становится эквивалентна первой.   

\item Докажите, что $\{0,1\}^*$ и множество корректных программ на каком-либо (любом) языке программирования 
разрешимо равномощны.

\textbf{Решение.} Построим две инъекции. Применим теорему Кантора-Берштейна. 

Обозначим $P$ - множество корректных программ. Тогда, можно определить следующие инъекции:

$f:P \rightarrow \{0,1\}^*$ - Бинарное представление скомпилированной программы в памяти. Инъективность 
очевидна.


$g:\{0,1\}^* \rightarrow P$ - Программа, состоящая из одной команды печати соответствующей последовательности 
нулей и единиц, которая указана в аргументах у команды печати. Тогда, если $g(a) = g(a') \Rightarrow $ 
выведется одна и та же строка $\Rightarrow a = a'$. 

Равномощность следует из теоремы Кантора-Бернштейна.

\item Определите множество простых чисел.

\textbf{Решение.} Воспользуемся \textit{separation axiom}. Тогда множество простых чисел можно определить так:
\begin{equation*}
	Prime = \{p \in \mathbb{N} \ | \ \forall k \leqslant \lfloor \sqrt{p} \rfloor (\gcd(p, k) = 1) \}
\end{equation*}

\item Определите следующие функции над $\mathbb{Q}$ и докажите их корректность: Рациональное число можно 
представить как пару $(p, q)$, где $p \in \mathbb{Z}, q \in \mathbb{N}$. 
\begin{enumerate}
\item Функция $neg : \mathbb{Q} \to \mathbb{Q}$, возвращающая обратное по сложению число.
\begin{equation*}
	neg = (p, q) \mapsto ((-p), q)
\end{equation*}

Корректность: Функция сохраняет эквивалентность: если $(x,y) \sim (x',y')$, то $neg([(x,y)]_\sim) 
\sim neg([(x',y')]_\sim)$. 

Это свойство выполнено: 

$neg([(x,y)]_\sim) = (-x, y)$, 

$neg([(x',y')]_\sim) = (-x', y')$. 

Заметим, что $(-x, y) \sim (-x', y') \Leftrightarrow -xy' = -x'y 
\Leftrightarrow xy' = x'y \Leftrightarrow (x,y) \sim (x',y')$. Показали, что эквивалентность 
сохранена.

\item Функция $inv : \mathbb{Q}_{\neq 0} \to \mathbb{Q}_{\neq 0}$, возвращающая обратное по умножению число.
\begin{equation*}
	inv = (p, q) \mapsto ((sign \ p) * q, abs \ p)
\end{equation*}

Корректность: Функция сохраняет эквивалентность: если $(x,y) \sim (x',y')$, то $inv([(x,y)]_\sim) 
\sim inv([(x',y')]_\sim)$. 

Это свойство выполнено: 

$inv([(x,y)]_\sim) = (sign(x) *y, x)$, 

$inv([(x',y')]_\sim) = (sign(x') * y', x')$. 

Заметим, что $sign(x) *y, x \sim sign(x') * y', x' \Leftrightarrow sign(x) * y * x' = (sign(x') * y' * x 
\Leftrightarrow x'y = xy' \Leftrightarrow (x,y) \sim (x',y')$. Показали, что эквивалентность сохранена.

\item Функция $plus : \mathbb{Q} \times \mathbb{Q} \to \mathbb{Q}$, возвращающая сумму двух чисел.
\begin{equation*}
	plus = ((p1, q1),  (p2, q2)) \mapsto (p1 * q2 + p2 * q1, q1 * q2)
\end{equation*}

Корректность: Функция сохраняет эквивалентность: если $(x_1,y_1) \sim (x_1',y_1')$ и $(x_2,y_2) \sim 
(x_2',y_2')$, то $plus([(x_1,y_1), (x_2, y_2)]_\sim) \sim plus([(x_1',y_1'), (x_2', y_2')]_\sim)$. 

Это свойство выполнено:

 $plus([(x_1,y_1), (x_2, y_2)]_\sim)     = (x_1 * y_2 + x_2 * y_1, y_1 * y_2)$,
 
 $plus([(x_1',y_1'), (x_2', y_2')]_\sim) = (x_1' * y_2' + x_2' * y_1', y_1' * y_2')$. 
 
 Заметим, что $(x_1 * y_2 + x_2 * y_1, y_1 * y_2) \sim (x_1' * y_2' + x_2' * y_1', y_1' * y_2') 
\Leftrightarrow (x_1 * y_2 + x_2 * y_1) * (y_1' * y_2') = (y_1 * y_2) * (x_1' * y_2' + x_2' * y_1') 
\Leftrightarrow x_1  y_2  y_1'  y_2' + x_2  y_1  y_1'  y_2' 
              = x_1'  y_2'  y_1  y_2 + x_2'  y_1'  y_1  y_2$
 
 Воспользуемся 
 $$(x_1,y_1) \sim (x_1',y_1') \Leftrightarrow x_1y_1' = x_1'y_1 = a$$
 $$(x_2,y_2) \sim (x_2',y_2') \Leftrightarrow x_2y_2' = x_2'y_2 = b$$
 
 Получим: 
 $$ y_2 y_2' \underline{y_1'x_1} +  y_1  y_1' \doubleunderline{x_2y_2'} 
 = \underline{x_1'y_1}  y_2' y_2 + \doubleunderline{x_2'y_2}  y_1'  y_1 \Leftrightarrow 
 a y_2 y_2' + b y_1  y_1' = a y_2' y_2 + b y_1'  y_1 $$
Последнее равенство очевидно, значит эквивалентность сохранена.

\end{enumerate}

\item Докажите, что существует биекция между двумя вариантами определения множества $\Pi (a \in A) B_a$, 
приведенных в лекции.

\textbf{Решение.} Приведём требуемую биекцию. 

Обозначим множество из первого определения $S_1 = \{f:A\rightarrow \Sigma(a\in A) B_a \ \big| \ \pi_1\circ f 
= id_A \}$

Множество из второго определения $S_2 \subset (A \rightarrow \cup_{a \in A} B_a)$

Определим $h:S_1 \rightarrow S_2$. 

$h(f) = a \mapsto \pi_2(f(a))$

$h^{-1}(g) = a \mapsto (a, g(a))$

Покажем, что это биекция:
\begin{align*}
	& h(h^{-1}(g)) = a \mapsto \pi_2(h^{-1}(g)(a)) = a \mapsto g(a) = g \\
	& h^{-1}(h(f)) = a \mapsto (a, h(f)(a)) = a \mapsto (a, \pi_2(f(a))) = a \mapsto f(a) = f
\end{align*}

\item \label{it:vec}
    Пусть $Vec(A,n)$ -- множество списков длины $n$, элементы которых лежат в множестве $A$. В лекции был 
    приведен пример функции $index$. Опишите аналогичным образом ``тип'' функций (то есть в каком множестве 
    они лежат, все эти множества будут множествами зависимых функций), приведенных ниже. Каждая из этих 
    функций должна принимать и возвращать элементы множеств вида $Vec(A,n)$ и, возможно, другие аргументы.
\begin{enumerate}
\item Функция $reverse$, разворачивающая список.

\textbf{Решение.} $$reverse : \Pi(n \in \mathbb{N}) (Vec(A, n) \rightarrow Vec(A, n)) $$

\item Функция $append$, конкатенирующая два списка.

\textbf{Решение.} $$append : \Pi((n,m) \in \mathbb{N}^2) (Vec(A, n) \rightarrow Vec(A, m) \rightarrow Vec(A, 
n + m)) $$

\item Функция $filter$, принимающая предикат и список длины $n$, и возвращающая список, с элементами 
исходного списка, которые удовлетворяют предикату.

\textbf{Решение.} $$filter : \Pi(n \in \mathbb{N}) (\{f:A\rightarrow \{\top, \bot\}\} \rightarrow Vec(A, n) 
\rightarrow \{v \in Vec(A, j) \ \big| \ j \in \mathbb{N}\cup {0}, j \leqslant n\}) $$
\end{enumerate}

\item Задания на хаскелле:
\begin{enumerate}
\item См. cb.hs.
\item Пусть $\mathbb{N}_{\geq 2} = \{ n \in \mathbb{N}\ |\ n \geq 2 \}$ и $m : \mathbb{N}_{\geq 2} \times 
\mathbb{N}_{\geq 2} \to \mathbb{N}$ вовзращает произведение чисел, то есть $m(x,y) = x \cdot y$.
    На лекции мы видели, что существует отношение эквивалентности $\sim$ на $\mathbb{N}_{\geq 2} \times 
    \mathbb{N}_{\geq 2}$, такое что $(\mathbb{N}_{\geq 2} \times \mathbb{N}_{\geq 2})/\!\!\sim$ равномощно 
    $im(m)$.
    
    Задайте тип на хаскелле, аналогичный $(\mathbb{N}_{\geq 2} \times \mathbb{N}_{\geq 2})/\!\!\sim$ (вам 
    понадобится задать $instance\ Eq$ для него).
    
    Определите биекцию на хаскелле между этим типом и $im(m)$.
\item
{
    Определите на хаскелле два варианта рациональных чисел: один через отношение эквивалентности, другой 
    через канонические представители.
}
    Определите биекцию между ними.
\end{enumerate}

\item Опциональная задача для любителей программирования с зависимыми типами. Реализуйте функции из задания 
\ref{it:vec} на агде.

\end{enumerate}

%\section*{Индукция}
\begin{enumerate}

\item Напишите нерекурсивное определение функции
\[ f(n) = \sum_{i < n} (f(i) + 1) \]

Докажите, используя (обобщенный) принцип индукции, равенство этих двух функций.

\textbf{Решение.} 

Это выражение образует последовательность вида: $0, 1, 3, 7, 15,...$ .Нерекурсивное 
определение функции: 
\begin{equation*}
f(n) = 2^n - 1
\end{equation*}

Докажем это равенство используя обобщенный принцип индукции.

Сначала заметим, что $f(0) = 2^0 - 1 = 1 - 1 = 0$. Верно.

Докажем индукционный переход. Пусть это верно для $f(n) = 2^n$. Покажем, что верно и для 
$f(S(n)) = f(n + 1)$.
\begin{equation*}
f(n + 1) = \sum_{i < n + 1} (f(i) + 1) = \sum_{i<n}(2^i) = 2^{n + 1} - 1.
\end{equation*}
Доказано.

\item Докажите, что принцип зависимой рекурсии эквивалентен принципам рекурсии и индукции.
    Hint: Чтобы доказать, что принцип индукции следует из принципа зависимой рекурсии, возьмите 
    в качестве $B$ следующую коллекцию:
    $B(n) = \{ * \}$, если верно $P(n)$, иначе $B(n) = \varnothing$.

\item Приведите контрпримеры, показывающие, что отдельно ни принципа рекурсии, ни принципа 
индукции не достаточно, чтобы гарантировать уникальность натуральных чисел.
    То есть нужно привести примеры множеств $\mathbb{N}_i$ вместе с $0_i \in \mathbb{N}_i$, $S_i 
    : \mathbb{N}_i \to \mathbb{N}_i$, где $i \in \{ 1, 2 \}$,
    таких что $\mathbb{N}_1$ удовлетворяет принципу рекурсии, $\mathbb{N}_2$ удовлетворяет 
    принципу индукции, но они не равномощны $\mathbb{N}$.
    
\textbf{Решение.} Выберем $\mathbb{N}_1 = \mathbb{N}_2 = \{0, 1\}, \ S_1(n) = S_2(n) = n\mod 2$. Для таких \textit{конечных} множеств принципы рекурсии и индукции по-отдельности справедливы, то 

\item Пусть $\mathbb{N}'$ -- некоторое множество вместе с $0' \in \mathbb{N}'$ и $S' : 
\mathbb{N}' \to \mathbb{N}'$.
    Тогда принцип $PM$ для $\mathbb{N}'$ говорит, что для любых $n,m \in \mathbb{N}$ если 
    $S'^n(0') = S'^m(0')$, то $n = m$.
\begin{itemize}
\item Докажите, что принцип рекурсии для $\mathbb{N}'$ эквивалентен $PM$.
\item Обратите внимание, что для доказательства этого факта нужно использовать принцип 
исключенного третьего. Укажите явно место, где оно необходимо.
\item Какое дополнительное предположение о $\mathbb{N}'$ нужно сделать, чтобы это доказательство 
работало без исключенного третьего?
\end{itemize}

\item Сформулируйте принципы рекурсии, индукции и зависимой рекурсии для множества $List(A)$.

\textbf{Решение.} Хоть и первые два принципа следуют из третьего, приведём все три.

\begin{itemize}
	\item Рекурсия.
	
	Для задания функций $f:List(A) \rightarrow B$ достаточно задать следующие данные:
	\begin{align*}
		&f(nil) = b \\
		&f(cons(a, xs))= e 
	\end{align*}
	где $b \in B$, $e$ - выражение, в котором может быть вызов $f(xs)$, и которое задаёт 
	элемент $B$.
	\item Индукция. Позволяет доказывать, что некоторый предикат $P(xs)$ выполняется на всех 
	элементах $xs \in List(A)$.
	
	Для этого достаточно показать, что верно $P(nil)$;
	
	А так же, $\forall xs \in List(A) : \left[ \left\{ P(xs) = \top \right\} \Rightarrow 
	\left\{ \forall a \in A : P(cons(a, xs)) = \top \right\} \right]$
	
	\item Зависимая рекурсия.
	
	Для задания функций $f:List(A) \rightarrow \Pi(xs \in List(A))B(xs)$ достаточно задать 
	следующие данные:
	\begin{align*}
	&f(nil) = b \\
	&f(cons(a, xs))= e 
	\end{align*}
	где $b \in B(nil)$, $e$ - выражение, в котором может быть вызов $f(xs) \in B(xs)$, и 
	которое задаёт элемент $(cons(a, xs) \in B(cons(a, xs))$.
\end{itemize}

\item Опишите индуктивным образом предикат на $\mathbb{N}$, задающий нечетные числа.

\textbf{Решение.}

\begin{center}
	\AxiomC{}
	\UnaryInfC{$1\ is\ odd$}
	\DisplayProof
	\qquad
	\AxiomC{$n\ is\ odd$}
	\UnaryInfC{$n + 2\ is\ odd$}
	\DisplayProof
\end{center}

\end{enumerate}


\begin{enumerate}

\item Опишите 2-сортную сигнатуру и теорию коммутативных колец с единицей и модулей над ними (определение этих понятий легко найти в интернете).

\item Рассмотрим сигнатуру $(\{N\}, \{ 0 : N, S : N \to N, + : N \times N \to N \})$.
    Рассмотрим следующую теорию:
\begin{align*}
0 + y & = y \\
S(x) + y & = S(x + y)
\end{align*}
Докажите, что следующие формулы невыводимы в этой теории
\begin{itemize}
\item $(x + y) + z = x + (y + z)$
\item $x + y = y + x$
\end{itemize}
Напомню, что для доказательства невыводимости формулы достаточно привести пример модели в которой эта формула не верна.

\item Рассмотрим сигнатуру $(\{D\}, \{ * : D \times D \to D, 1 : D, f : D \to D, g : D \to D, i_1 : D \to D, i_2 : D \to D \})$.
    Рассмотрим следующую теорию в ней:
\begin{align*}
(x * y) * z & = x * (y * z) \\
x * 1 & = x \\
1 * x & = x \\
f(f(x)) & = f(x) \\
g(g(x)) & = g(x) \\
f(g(x)) & = g(f(x)) \\
i_1(f(x)) * g(x) & = 1 \\
f(x) * i_2(g(x)) & = 1
\end{align*}
Какие из следующих утверждений являются теоремами этой теории? Докажите это.
\begin{enumerate}
\item $i_1(x) = i_2(x)$
\item $i_1(x) * x = 1$
\item $f(x) = g(x)$
\item $f(x) = x$
\end{enumerate}
При доказательстве выводимости можно опускать очевидные шаги, такие как применения ассоциативности и аксиом $1 * x = x$ и $x * 1 = x$.

\item Рассмотрим сигнатуру $(\{D\}, \{ * : D \times D \to D, + : D \times D \to D, 1 : D, 0 : D, - : D \to D \})$.
    Теория колец с единицей выглядит следующим образом:
\begin{align*}
(x + y) + z & = x + (y + z) \\
x + 0 & = x \\
0 + x & = x \\
x + y & = y + x \\
x + -x & = 0 \\
(x * y) * z & = x * (y * z) \\
x * 1 & = x \\
1 * x & = x \\
x * (y + z) & = (x * y) + (x * z) \\
(y + z) * x & = (y * x) + (z * x)
\end{align*}
Добавим к этой теории следующую аксиому:
\[ x * x = x \]
Докажите, что в этой расширенной теории выводимы следующие формулы:
\begin{enumerate}
\item $x * y = y * x$
\item $x + x = 0$
\end{enumerate}

\end{enumerate}


\end{document}
