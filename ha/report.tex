\documentclass[fleqn]{article}
\usepackage{cmap}
\usepackage[left=1in, right=1in, top=1in, bottom=1in]{geometry}
\usepackage{mathexam}
\usepackage{mathtext} 				% русские буквы в фомулах
\usepackage[T2A]{fontenc}			% кодировка
\usepackage[utf8]{inputenc}			% кодировка исходного текста
\usepackage[english,russian]{babel}	% локализация и переносы
\usepackage{enumerate}
%%% Дополнительная работа с математикой
\usepackage{amsmath,amsfonts,amssymb,amsthm,mathtools,amsthm} % AMS
\usepackage{icomma} % "Умная" запятая: $0,2$ --- число, $0, 2$ --- перечисление
\usepackage{graphicx}
\usepackage{longtable}

% Двойные квадратные скобки
\usepackage{stmaryrd}
\newcommand{\llb}{\llbracket}
\newcommand{\rrb}{\rrbracket}

%% Шрифты
\usepackage{euscript}	 % Шрифт Евклид
\usepackage{mathrsfs} % Красивый матшрифтw
\ExamClass{SE Academic University}
\ExamName{All homeworks}
\ExamHead{\today}

%% Шрифты 
\usepackage{euscript}	 % Шрифт Евклид
\usepackage{mathrsfs} % Красивый матшрифт

%Листинг кода
%\usepackage{listings}
\usepackage{listingsutf8}
\usepackage{color}
\usepackage{bussproofs}
\renewcommand{\qedsymbol}{$\blacksquare$}
%Для листинга кода
\definecolor{mygreen}{rgb}{0,0.6,0}
\definecolor{mygray}{rgb}{0.5,0.5,0.5}
\definecolor{mymauve}{rgb}{0.63,0.082,0.082}


\lstset{
	inputencoding=utf8,
	%
	backgroundcolor=\color{white},   % choose the background color; you must add \usepackage{color} or \usepackage{xcolor}
	basicstyle=\footnotesize,        % the size of the fonts that are used for the code
	breakatwhitespace=false,         % sets if automatic breaks should only happen at whitespace
	breaklines=true,                 % sets automatic line breaking
	captionpos=b,                    % sets the caption-position to bottom
	commentstyle=\color{black},    % comment style
	deletekeywords={...},            % if you want to delete keywords from the given language
	escapeinside={\%*}{*)},          % if you want to add LaTeX within your code
	extendedchars=\true,              % lets you use non-ASCII characters; for 8-bits encodings only, does not work with UTF-8
	frame=false,                    % adds a frame around the code
	keepspaces=true,                 % keeps spaces in text, useful for keeping indentation of code (possibly needs columns=flexible)
	keywordstyle=\color{blue},       % keyword style
	morekeywords={*,...},            % if you want to add more keywords to the set
	numbers=left,                    % where to put the line-numbers; possible values are (none, left, right)
	numbersep=5pt,                   % how far the line-numbers are from the code
	numberstyle=\tiny\color{black}, % the style that is used for the line-numbers
	rulecolor=\color{white},         % if not set, the frame-color may be changed on line-breaks within not-black text (e.g. comments (green here))
	showspaces=false,                % show spaces everywhere adding particular underscores; it overrides 'showstringspaces'
	showstringspaces=false,          % underline spaces within strings only
	showtabs=false,                  % show tabs within strings adding particular underscores
	stepnumber=1,                    % the step between two line-numbers. If it's 1, each line will be numbered
	stringstyle=\color{black},     % string literal style
	tabsize=4                  % sets default tabsize to 2 spaces
	% show the filename of files included with \lstinputlisting; also try caption instead of title
}   


\let\ds\displaystyle

\begin{document}

\section{Множества}
\begin{enumerate}

\item Определите множество частичных функций через множество их графиков.

\textbf{Решение.} 

Множество подмножеств $G \subseteq A \times (B \cup \{undefined\})$,таких что для любого $a \in A$ существует 
единственный $b \in (B \cup \{undefined\})$, такой что $(a, b) \in G$.

\item Композиция функций $f : A \to B$ и $g : B \to C$ -- это функция $g \circ f : A \to C$, такая что $(g 
\circ f)(a) = g(f(a))$.
    В хаскелле есть аналог этой операции -- это $g\ .\ f$. Задайте график функции $g \circ f$ через графики 
    $f$ и $g$.

\textbf{Решение.} 

Рассмотрим графики функций $f$ и $g$. Обозначим $G_f \subseteq A \times B$, $G_g \subseteq B \times C$. 
Обозначим $B_f$ множество элементов $b \in B \exists a \in A : (a, b) \in G_f$. А в качестве $B_g$ обозначим 
множество элементов $b\in B : \exists c \in C : (b, c) \in G_g$. Тогда введём $B = B_f \cap B_g$. Тогда в 
качестве графика функции $g \circ f$ можно взять множества пар $(a, c)$ таких, что $\exists b \in B : (a, b) 
\in G_f \land (b, c) \in G_g$.  

\item Докажите, что если $A$ вкладывается в $B$ и $B$ вкладывается в $C$, то $A$ вкладывается в $C$.

\textbf{Решение.} 

$A$ вкладывается в $B$ $\Rightarrow \exists f:A\rightarrow B - $ инъекция.

$B$ вкладывается в $C$ $\Rightarrow \exists g:B\rightarrow C - $ инъекция.

Покажем, что композиция $g \circ f : A\rightarrow C - $ является инъекцией.

Для этого достаточно показать, что если $(g \circ f)(a) = (g \circ f)(a')$ то $a = a'$. Заметим, что 
$g:B\rightarrow C$ - инъекция, это значит, что $f(a) = f(a')$. $f:A\rightarrow C$ - инъекция, значит $a = a'$.

\item Докажите, что если $A$ накрывает $B$ и $B$ накрывает $C$, то $A$ накрывает $C$.

\textbf{Решение.} 

$A$ накрывает в $B$ $\Rightarrow \exists f:A\rightarrow B - $ сюръекция.

$B$ накрывает в $C$ $\Rightarrow \exists g:B\rightarrow C - $ сюръекция.

Покажем, что композиция $g \circ f : A\rightarrow C - $ является сюръекцией.

Для этого достаточно показать, что $\forall c \in C \exists a \in A : (g \circ f)(a) = c$. заметим, что $g$ - 
сюръекция, значит, $\forall c \exists b \in B : g(b) = c$. Теперь, заметим, что $f$ - сюръекция, значит, 
$\forall b \in B \exists a \in A : f(a) = b$. Объединяя эти два утверждения, получим определение сюръекции 
для $(g \circ f)$. $\forall c \in C \exists b \in B : g(b) = c \exists a \in A : f(a) = b \Rightarrow \forall 
c \in C \exists a \in A : g(f(a)) = c$. Значит композиция сюръекций снова сюръекция.

\item Докажите, что следующие свойства верны:
\begin{enumerate}
\item $A$ равномощно $A$.

\textbf{Решение.} 

Существует тождественное отображение $id: A\rightarrow A$, $id(a) = a$. Т.к. это биекция(для $a\neq \hat{a}$, 
они переходят $id(a) = a \neq \hat{a} = id(\hat{a}),$ и $\forall a\in A \exists\hat{a} = a \in A : 
id(\hat{a}) = a$ ), то $A$ равномощно $A$. 

\item Если $A$ равномощно $B$, то $B$ равномощно $A$.

\textbf{Решение.} 

$A$ равномощно $B$ $\Rightarrow$ $\exists \varphi(a):A\rightarrow B$ - биекция $\Leftrightarrow$ $\exists 
\varphi^{-1} : B \rightarrow A$ - обратная функция - биекция $\Rightarrow$ $B$ равномощно $A$ 

\item Если $A$ равномощно $B$ и $B$ равномощно $C$, то $A$ равномощно $C$.

\textbf{Решение.} 

Из такой постановки задачи следуют постановки задач пунктах $3, 4$. В них мы нашли функцию $\varphi = g \circ 
f$, которая является и накрытием и вложением $A$ в $C$, а значит, инъекцией и сюръекцией $\Rightarrow$ 
$\varphi$ является биекцией $\Rightarrow$ $A$ равномощно $C$.

\end{enumerate}

\item Докажите, что множества $A \amalg B \to C$ и $(A \to C) \times (B \to C)$ равномощны.

\textbf{Решение.} 

Приведем биекцию $\varphi : (A \amalg B \to C) \rightarrow ((A \to C) \times (B \to C))$

\begin{equation*}
	\varphi(f) = (a \mapsto f(Left(a)), b \mapsto f(Right(b)))
\end{equation*}

и обратная к ней

\begin{align*}
	& \varphi^{-1}(g) = Left(x) \mapsto \pi_1(g)(x) \\
	& \varphi^{-1}(g) = Right(x) \mapsto \pi_2(g)(x)
\end{align*}

Несложно убедиться, что эти функции взаимно обратные.

Выберем произвольную функцию $f\in A \amalg B \to C$:
\begin{align*}
	& f(Left(x)) = e \\
	& f(Right(x)) = d
\end{align*}

и вычислим значение $\varphi^{-1}(\varphi(f))$:

\begin{align*}
	& \varphi^{-1}(\varphi(f)) = \pi_1(a \mapsto f(Left(a)), b \mapsto 	f(Right(b))) = f(Left(x)) \\
	& \varphi^{-1}(\varphi(f)) = \pi_2(a \mapsto f(Left(a)), b \mapsto 	f(Right(b))) = f(Right(x))
\end{align*}

Так же несложно показать и обратное преобразование:

\begin{equation*}
\begin{array} {lcl}
	\varphi(\varphi^{-1}(g)) = (a \mapsto \varphi^{-1}(g)(Left(a)), b 	\mapsto \varphi^{-1}(g)(Right(b))) = 
	\\
	= (a \mapsto \pi_1(g)(Left(a)), b \mapsto \pi_2(g)(Right(b))) = (\pi_1(g), \pi_2(g)) = g
\end{array}
\end{equation*}

Таким образом, $\varphi$ - биекция, и множества $A \amalg B \to C$ и $(A \to C) \times (B \to C)$ равномощны.

\item Докажите, что множества $A \times B \to C$ и $A \to (B \to C)$ равномощны.

\textbf{Решение.} 

По аналогии с предыдущим заданием придумаем биекцию $\varphi : (A \times B \to C)\to (A \to (B \to C))$.

\begin{equation*}
	\varphi(f) = x \mapsto (y \mapsto f(x, y))
\end{equation*}

Можно заметить, что получили каррирование функции $f \in A \times B \to C$.

Обратная функция к $\varphi(f)$:

\begin{equation*}
	\varphi^{-1}(g) = p \mapsto g(\pi_1(p))(\pi_2(p))
\end{equation*}

Таким образом, $\varphi$ - биекция, и множества $A \times B \to C$ и $A \to (B \to C)$ равномощны.

\item Докажите, что $|\mathcal{P}(A)| = 2^{|A|}$ и $|A \to B| = |B|^{|A|}$.

\textbf{Решение.} 

Очевидно, первое равенство является прямым следствием второго, т.к. $\mathcal{P}(A)$ имеет 
взаимно-однозначное соответствие с отображениями $A \to \Omega$, где $\Omega = \{ \top, \bot \}$, значит 
$|\mathcal{P}(A)| = |A \to \Omega| = |\Omega|^{|A|} = 2^{|A|}$.

Докажем второе равенство.

Для этого достаточно найти биекцию $A \to B$ на множество 

$\{0, 1, ..., |B|^{|A|} - 1 \}$.

Раз уж $A$ и $B$ конечные, то из элементы можно пронумеровать номерами $\{0, 1, 2,..., |A|\}$ и $\{0, 1, 
2,..., |B|\}$ соответственно. Теперь, если записать последовательно $A$ образов элементов множества $A$ для 
которой $f \in A \to B$, то получим последовательность $f(a_{|A|}), f(a_{|A| - 1}), f(a_{|A| - 2}), .., 
f(a_2), f(a_1)$ элементов из $B$. Их можно отобразить на множество $\{0, 1, 2,..., |B|\}$ Получим число в 
системе счисления с основанием равным $|B|$ и $|A|$ разрядами. Переведя это число по соответствующим правилам 
в десятичную систему исчиления получим значение из интервала $[0; |B|^{|A|} - 1]$. Так же есть возможность 
совершить обратное преобразование (последовательное взятие остатка по модулю $|B|$ и целочисленное деление на 
значение $|B|$). Таким образом, требуемая биекция найдена, следовательно $|A \to B| = |B|^{|A|}$.

\end{enumerate}

\section{Конструкции над множествами}

\def\doubleunderline#1{\underline{\underline{#1}}}

Мы будем говорить, что два множества разрешимо равномощны, если у этих множеств есть аналоги в виде типов 
языка хаскелл, и между этими типами существуют взаимно обратные функции языка хаскелл. Аналогом множества 
$\mathbb{N}$ является тип $data\ Nat\ = Zero\ |\ Suc\ Nat$. Аналогом множества $\{0,1\}^*$ является тип 
$[Bool]$. Если $X$ -- подмножество $\{0,1\}^*$, то его аналогом является тип $type X = [Bool]$, но 
предполагается, что в функции $X \to a$ не передаются аргументы, выходящие за пределы $X$, и функции $a \to 
X$ не возвращают результат, выходящий за его пределы.

\begin{enumerate}

\item Докажите, что $\mathbb{N}$ и $\{0,1\}^*$ разрешимо равномощны, где второе множество -- это множество 
последовательностей из 0 и 1.

\textbf{Решение.} Построим две инъекции. Применим теорему Кантора-Берштейна. 

$f:\mathbb{N} \rightarrow \{0, 1\}*$. - представление числа в двоичной системе исчисления. Инъективность 
очевидна.

$g:\{0,1\}^* \rightarrow \mathbb{N}$.- добавим к каждому члену последовательности из $0$ и $1$ по единице, 
получим число в троичной системе исчисления без ведущих нулей. Переведем его в десятичную систему исчисление. 
Инъективность следует из отсутствия ведущих нулей и биекцией между троичной и десятичной системами 
исчисления. 

Равномощность следует из теоремы Кантора-Бернштейна.

\item Докажите, что $\{0,1\}^*$ и $\mathbb{N}_2$ разрешимо равномощны, где второе множество -- это множество 
двоичных натуральных чисел, то есть последовательностей 0 и 1 без ведущих нулей (кроме случая, когда     
последовательность состоит из одной цифры).
    
\textbf{Решение.} Заметим, что между $\mathbb{N}_2$ и $\mathbb{N}$ существует биекция - перевод числа в 
двоичную систему и обратно, то задача становится эквивалентна первой.   

\item Докажите, что $\{0,1\}^*$ и множество корректных программ на каком-либо (любом) языке программирования 
разрешимо равномощны.

\textbf{Решение.} Построим две инъекции. Применим теорему Кантора-Берштейна. 

Обозначим $P$ - множество корректных программ. Тогда, можно определить следующие инъекции:

$f:P \rightarrow \{0,1\}^*$ - Бинарное представление скомпилированной программы в памяти. Инъективность 
очевидна.


$g:\{0,1\}^* \rightarrow P$ - Программа, состоящая из одной команды печати соответствующей последовательности 
нулей и единиц, которая указана в аргументах у команды печати. Тогда, если $g(a) = g(a') \Rightarrow $ 
выведется одна и та же строка $\Rightarrow a = a'$. 

Равномощность следует из теоремы Кантора-Бернштейна.

\item Определите множество простых чисел.

\textbf{Решение.} Воспользуемся \textit{separation axiom}. Тогда множество простых чисел можно определить так:
\begin{equation*}
	Prime = \{p \in \mathbb{N} \ | \ \forall k \leqslant \lfloor \sqrt{p} \rfloor (\gcd(p, k) = 1) \}
\end{equation*}

\item Определите следующие функции над $\mathbb{Q}$ и докажите их корректность: Рациональное число можно 
представить как пару $(p, q)$, где $p \in \mathbb{Z}, q \in \mathbb{N}$. 
\begin{enumerate}
\item Функция $neg : \mathbb{Q} \to \mathbb{Q}$, возвращающая обратное по сложению число.
\begin{equation*}
	neg = (p, q) \mapsto ((-p), q)
\end{equation*}

Корректность: Функция сохраняет эквивалентность: если $(x,y) \sim (x',y')$, то $neg([(x,y)]_\sim) 
\sim neg([(x',y')]_\sim)$. 

Это свойство выполнено: 

$neg([(x,y)]_\sim) = (-x, y)$, 

$neg([(x',y')]_\sim) = (-x', y')$. 

Заметим, что $(-x, y) \sim (-x', y') \Leftrightarrow -xy' = -x'y 
\Leftrightarrow xy' = x'y \Leftrightarrow (x,y) \sim (x',y')$. Показали, что эквивалентность 
сохранена.

\item Функция $inv : \mathbb{Q}_{\neq 0} \to \mathbb{Q}_{\neq 0}$, возвращающая обратное по умножению число.
\begin{equation*}
	inv = (p, q) \mapsto ((sign \ p) * q, abs \ p)
\end{equation*}

Корректность: Функция сохраняет эквивалентность: если $(x,y) \sim (x',y')$, то $inv([(x,y)]_\sim) 
\sim inv([(x',y')]_\sim)$. 

Это свойство выполнено: 

$inv([(x,y)]_\sim) = (sign(x) *y, x)$, 

$inv([(x',y')]_\sim) = (sign(x') * y', x')$. 

Заметим, что $sign(x) *y, x \sim sign(x') * y', x' \Leftrightarrow sign(x) * y * x' = (sign(x') * y' * x 
\Leftrightarrow x'y = xy' \Leftrightarrow (x,y) \sim (x',y')$. Показали, что эквивалентность сохранена.

\item Функция $plus : \mathbb{Q} \times \mathbb{Q} \to \mathbb{Q}$, возвращающая сумму двух чисел.
\begin{equation*}
	plus = ((p1, q1),  (p2, q2)) \mapsto (p1 * q2 + p2 * q1, q1 * q2)
\end{equation*}

Корректность: Функция сохраняет эквивалентность: если $(x_1,y_1) \sim (x_1',y_1')$ и $(x_2,y_2) \sim 
(x_2',y_2')$, то $plus([(x_1,y_1), (x_2, y_2)]_\sim) \sim plus([(x_1',y_1'), (x_2', y_2')]_\sim)$. 

Это свойство выполнено:

 $plus([(x_1,y_1), (x_2, y_2)]_\sim)     = (x_1 * y_2 + x_2 * y_1, y_1 * y_2)$,
 
 $plus([(x_1',y_1'), (x_2', y_2')]_\sim) = (x_1' * y_2' + x_2' * y_1', y_1' * y_2')$. 
 
 Заметим, что $(x_1 * y_2 + x_2 * y_1, y_1 * y_2) \sim (x_1' * y_2' + x_2' * y_1', y_1' * y_2') 
\Leftrightarrow (x_1 * y_2 + x_2 * y_1) * (y_1' * y_2') = (y_1 * y_2) * (x_1' * y_2' + x_2' * y_1') 
\Leftrightarrow x_1  y_2  y_1'  y_2' + x_2  y_1  y_1'  y_2' 
              = x_1'  y_2'  y_1  y_2 + x_2'  y_1'  y_1  y_2$
 
 Воспользуемся 
 $$(x_1,y_1) \sim (x_1',y_1') \Leftrightarrow x_1y_1' = x_1'y_1 = a$$
 $$(x_2,y_2) \sim (x_2',y_2') \Leftrightarrow x_2y_2' = x_2'y_2 = b$$
 
 Получим: 
 $$ y_2 y_2' \underline{y_1'x_1} +  y_1  y_1' \doubleunderline{x_2y_2'} 
 = \underline{x_1'y_1}  y_2' y_2 + \doubleunderline{x_2'y_2}  y_1'  y_1 \Leftrightarrow 
 a y_2 y_2' + b y_1  y_1' = a y_2' y_2 + b y_1'  y_1 $$
Последнее равенство очевидно, значит эквивалентность сохранена.

\end{enumerate}

\item Докажите, что существует биекция между двумя вариантами определения множества $\Pi (a \in A) B_a$, 
приведенных в лекции.

\textbf{Решение.} Приведём требуемую биекцию. 

Обозначим множество из первого определения $S_1 = \{f:A\rightarrow \Sigma(a\in A) B_a \ \big| \ \pi_1\circ f 
= id_A \}$

Множество из второго определения $S_2 \subset (A \rightarrow \cup_{a \in A} B_a)$

Определим $h:S_1 \rightarrow S_2$. 

$h(f) = a \mapsto \pi_2(f(a))$

$h^{-1}(g) = a \mapsto (a, g(a))$

Покажем, что это биекция:
\begin{align*}
	& h(h^{-1}(g)) = a \mapsto \pi_2(h^{-1}(g)(a)) = a \mapsto g(a) = g \\
	& h^{-1}(h(f)) = a \mapsto (a, h(f)(a)) = a \mapsto (a, \pi_2(f(a))) = a \mapsto f(a) = f
\end{align*}

\item \label{it:vec}
    Пусть $Vec(A,n)$ -- множество списков длины $n$, элементы которых лежат в множестве $A$. В лекции был 
    приведен пример функции $index$. Опишите аналогичным образом ``тип'' функций (то есть в каком множестве 
    они лежат, все эти множества будут множествами зависимых функций), приведенных ниже. Каждая из этих 
    функций должна принимать и возвращать элементы множеств вида $Vec(A,n)$ и, возможно, другие аргументы.
\begin{enumerate}
\item Функция $reverse$, разворачивающая список.

\textbf{Решение.} $$reverse : \Pi(n \in \mathbb{N}) (Vec(A, n) \rightarrow Vec(A, n)) $$

\item Функция $append$, конкатенирующая два списка.

\textbf{Решение.} $$append : \Pi((n,m) \in \mathbb{N}^2) (Vec(A, n) \rightarrow Vec(A, m) \rightarrow Vec(A, 
n + m)) $$

\item Функция $filter$, принимающая предикат и список длины $n$, и возвращающая список, с элементами 
исходного списка, которые удовлетворяют предикату.

\textbf{Решение.} $$filter : \Pi(n \in \mathbb{N}) (\{f:A\rightarrow \{\top, \bot\}\} \rightarrow Vec(A, n) 
\rightarrow \{v \in Vec(A, j) \ \big| \ j \in \mathbb{N}\cup {0}, j \leqslant n\}) $$
\end{enumerate}

\item Задания на хаскелле:
\begin{enumerate}
\item См. cb.hs.
\item Пусть $\mathbb{N}_{\geq 2} = \{ n \in \mathbb{N}\ |\ n \geq 2 \}$ и $m : \mathbb{N}_{\geq 2} \times 
\mathbb{N}_{\geq 2} \to \mathbb{N}$ вовзращает произведение чисел, то есть $m(x,y) = x \cdot y$.
    На лекции мы видели, что существует отношение эквивалентности $\sim$ на $\mathbb{N}_{\geq 2} \times 
    \mathbb{N}_{\geq 2}$, такое что $(\mathbb{N}_{\geq 2} \times \mathbb{N}_{\geq 2})/\!\!\sim$ равномощно 
    $im(m)$.
    
    Задайте тип на хаскелле, аналогичный $(\mathbb{N}_{\geq 2} \times \mathbb{N}_{\geq 2})/\!\!\sim$ (вам 
    понадобится задать $instance\ Eq$ для него).
    
    Определите биекцию на хаскелле между этим типом и $im(m)$.
\item
{
    Определите на хаскелле два варианта рациональных чисел: один через отношение эквивалентности, другой 
    через канонические представители.
}
    Определите биекцию между ними.
\end{enumerate}

\item Опциональная задача для любителей программирования с зависимыми типами. Реализуйте функции из задания 
\ref{it:vec} на агде.

\end{enumerate}

\section*{Индукция}
\begin{enumerate}

\item Напишите нерекурсивное определение функции
\[ f(n) = \sum_{i < n} (f(i) + 1) \]

Докажите, используя (обобщенный) принцип индукции, равенство этих двух функций.

\textbf{Решение.} 

Это выражение образует последовательность вида: $0, 1, 3, 7, 15,...$ .Нерекурсивное 
определение функции: 
\begin{equation*}
f(n) = 2^n - 1
\end{equation*}

Докажем это равенство используя обобщенный принцип индукции.

Сначала заметим, что $f(0) = 2^0 - 1 = 1 - 1 = 0$. Верно.

Докажем индукционный переход. Пусть это верно для $f(n) = 2^n$. Покажем, что верно и для 
$f(S(n)) = f(n + 1)$.
\begin{equation*}
f(n + 1) = \sum_{i < n + 1} (f(i) + 1) = \sum_{i<n}(2^i) = 2^{n + 1} - 1.
\end{equation*}
Доказано.

\item Докажите, что принцип зависимой рекурсии эквивалентен принципам рекурсии и индукции.
    Hint: Чтобы доказать, что принцип индукции следует из принципа зависимой рекурсии, возьмите 
    в качестве $B$ следующую коллекцию:
    $B(n) = \{ * \}$, если верно $P(n)$, иначе $B(n) = \varnothing$.

\item Приведите контрпримеры, показывающие, что отдельно ни принципа рекурсии, ни принципа 
индукции не достаточно, чтобы гарантировать уникальность натуральных чисел.
    То есть нужно привести примеры множеств $\mathbb{N}_i$ вместе с $0_i \in \mathbb{N}_i$, $S_i 
    : \mathbb{N}_i \to \mathbb{N}_i$, где $i \in \{ 1, 2 \}$,
    таких что $\mathbb{N}_1$ удовлетворяет принципу рекурсии, $\mathbb{N}_2$ удовлетворяет 
    принципу индукции, но они не равномощны $\mathbb{N}$.
    
\textbf{Решение.} Выберем $\mathbb{N}_1 = \mathbb{N}_2 = \{0, 1\}, \ S_1(n) = S_2(n) = n\mod 2$. Для таких \textit{конечных} множеств принципы рекурсии и индукции по-отдельности справедливы, то 

\item Пусть $\mathbb{N}'$ -- некоторое множество вместе с $0' \in \mathbb{N}'$ и $S' : 
\mathbb{N}' \to \mathbb{N}'$.
    Тогда принцип $PM$ для $\mathbb{N}'$ говорит, что для любых $n,m \in \mathbb{N}$ если 
    $S'^n(0') = S'^m(0')$, то $n = m$.
\begin{itemize}
\item Докажите, что принцип рекурсии для $\mathbb{N}'$ эквивалентен $PM$.
\item Обратите внимание, что для доказательства этого факта нужно использовать принцип 
исключенного третьего. Укажите явно место, где оно необходимо.
\item Какое дополнительное предположение о $\mathbb{N}'$ нужно сделать, чтобы это доказательство 
работало без исключенного третьего?
\end{itemize}

\item Сформулируйте принципы рекурсии, индукции и зависимой рекурсии для множества $List(A)$.

\textbf{Решение.} Хоть и первые два принципа следуют из третьего, приведём все три.

\begin{itemize}
	\item Рекурсия.
	
	Для задания функций $f:List(A) \rightarrow B$ достаточно задать следующие данные:
	\begin{align*}
		&f(nil) = b \\
		&f(cons(a, xs))= e 
	\end{align*}
	где $b \in B$, $e$ - выражение, в котором может быть вызов $f(xs)$, и которое задаёт 
	элемент $B$.
	\item Индукция. Позволяет доказывать, что некоторый предикат $P(xs)$ выполняется на всех 
	элементах $xs \in List(A)$.
	
	Для этого достаточно показать, что верно $P(nil)$;
	
	А так же, $\forall xs \in List(A) : \left[ \left\{ P(xs) = \top \right\} \Rightarrow 
	\left\{ \forall a \in A : P(cons(a, xs)) = \top \right\} \right]$
	
	\item Зависимая рекурсия.
	
	Для задания функций $f:List(A) \rightarrow \Pi(xs \in List(A))B(xs)$ достаточно задать 
	следующие данные:
	\begin{align*}
	&f(nil) = b \\
	&f(cons(a, xs))= e 
	\end{align*}
	где $b \in B(nil)$, $e$ - выражение, в котором может быть вызов $f(xs) \in B(xs)$, и 
	которое задаёт элемент $(cons(a, xs) \in B(cons(a, xs))$.
\end{itemize}

\item Опишите индуктивным образом предикат на $\mathbb{N}$, задающий нечетные числа.

\textbf{Решение.}

\begin{center}
	\AxiomC{}
	\UnaryInfC{$1\ is\ odd$}
	\DisplayProof
	\qquad
	\AxiomC{$n\ is\ odd$}
	\UnaryInfC{$n + 2\ is\ odd$}
	\DisplayProof
\end{center}

\end{enumerate}


\begin{enumerate}

\item Опишите 2-сортную сигнатуру и теорию коммутативных колец с единицей и модулей над ними (определение этих понятий легко найти в интернете).

\item Рассмотрим сигнатуру $(\{N\}, \{ 0 : N, S : N \to N, + : N \times N \to N \})$.
    Рассмотрим следующую теорию:
\begin{align*}
0 + y & = y \\
S(x) + y & = S(x + y)
\end{align*}
Докажите, что следующие формулы невыводимы в этой теории
\begin{itemize}
\item $(x + y) + z = x + (y + z)$
\item $x + y = y + x$
\end{itemize}
Напомню, что для доказательства невыводимости формулы достаточно привести пример модели в которой эта формула не верна.

\item Рассмотрим сигнатуру $(\{D\}, \{ * : D \times D \to D, 1 : D, f : D \to D, g : D \to D, i_1 : D \to D, i_2 : D \to D \})$.
    Рассмотрим следующую теорию в ней:
\begin{align*}
(x * y) * z & = x * (y * z) \\
x * 1 & = x \\
1 * x & = x \\
f(f(x)) & = f(x) \\
g(g(x)) & = g(x) \\
f(g(x)) & = g(f(x)) \\
i_1(f(x)) * g(x) & = 1 \\
f(x) * i_2(g(x)) & = 1
\end{align*}
Какие из следующих утверждений являются теоремами этой теории? Докажите это.
\begin{enumerate}
\item $i_1(x) = i_2(x)$
\item $i_1(x) * x = 1$
\item $f(x) = g(x)$
\item $f(x) = x$
\end{enumerate}
При доказательстве выводимости можно опускать очевидные шаги, такие как применения ассоциативности и аксиом $1 * x = x$ и $x * 1 = x$.

\item Рассмотрим сигнатуру $(\{D\}, \{ * : D \times D \to D, + : D \times D \to D, 1 : D, 0 : D, - : D \to D \})$.
    Теория колец с единицей выглядит следующим образом:
\begin{align*}
(x + y) + z & = x + (y + z) \\
x + 0 & = x \\
0 + x & = x \\
x + y & = y + x \\
x + -x & = 0 \\
(x * y) * z & = x * (y * z) \\
x * 1 & = x \\
1 * x & = x \\
x * (y + z) & = (x * y) + (x * z) \\
(y + z) * x & = (y * x) + (z * x)
\end{align*}
Добавим к этой теории следующую аксиому:
\[ x * x = x \]
Докажите, что в этой расширенной теории выводимы следующие формулы:
\begin{enumerate}
\item $x * y = y * x$
\item $x + x = 0$
\end{enumerate}

\end{enumerate}


\section*{Пропозициональная логика}

Множество типов лямбда исчисления:
\begin{center}
\AxiomC{}
\RightLabel{, $P \in Var$}
\UnaryInfC{$P \in Type$}
\DisplayProof
\qquad
\AxiomC{}
\UnaryInfC{$\bot \in Type$}
\DisplayProof
\qquad
\AxiomC{$\varphi \in Type$}
\AxiomC{$\psi \in Type$}
\BinaryInfC{$\varphi \to \psi \in Type$}
\DisplayProof
\end{center}

\begin{center}
\AxiomC{$\varphi \in Type$}
\AxiomC{$\psi \in Type$}
\BinaryInfC{$\varphi \times \psi \in Type$}
\DisplayProof
\qquad
\AxiomC{$\varphi \in Type$}
\AxiomC{$\psi \in Type$}
\BinaryInfC{$\varphi \amalg \psi \in Type$}
\DisplayProof
\end{center}

Множество предтермов лямбда исчисления (мы используем разные множества для переменных в типах и переменных в 
термах):
\begin{center}
\AxiomC{}
\RightLabel{, $x \in Var'$}
\UnaryInfC{$x \in Term$}
\DisplayProof
\qquad
\AxiomC{$t \in Term$}
\RightLabel{, $\varphi \in Type$}
\UnaryInfC{$absurd_\varphi\,t \in Term$}
\DisplayProof
\qquad
\end{center}

\begin{center}
\AxiomC{$t \in Term$}
\RightLabel{, $x \in Var'$}
\UnaryInfC{$\lambda x.\,t \in Term$}
\DisplayProof
\qquad
\AxiomC{$t \in Term$}
\AxiomC{$t' \in Term$}
\BinaryInfC{$t\,t' \in Term$}
\DisplayProof
\end{center}

\begin{center}
\AxiomC{$a \in Term$}
\AxiomC{$b \in Term$}
\BinaryInfC{$(a,b) \in Term$}
\DisplayProof
\qquad
\AxiomC{$t \in Term$}
\UnaryInfC{$fst\,t \in Term$}
\DisplayProof
\qquad
\AxiomC{$t \in Term$}
\UnaryInfC{$snd\,t \in Term$}
\DisplayProof
\end{center}

\begin{center}
\AxiomC{$t \in Term$}
\UnaryInfC{$Left\,t \in Term$}
\DisplayProof
\qquad
\AxiomC{$t \in Term$}
\UnaryInfC{$Right\,t \in Term$}
\DisplayProof
\end{center}

\begin{center}
\AxiomC{$e \in Term$}
\AxiomC{$a \in Term$}
\AxiomC{$b \in Term$}
\RightLabel{, $x,y \in Var'$}
\TrinaryInfC{$\mathbf{case}\,e\,\mathbf{of}\,\{\,Left(x) \to a; Right(y) \to b\,\} \in Term$}
\DisplayProof
\end{center}

Правила типизации:
\begin{center}
\AxiomC{}
\RightLabel{, $(x : \varphi) \in \Gamma$}
\UnaryInfC{$\Gamma \vdash x : \varphi$}
\DisplayProof
\qquad
\AxiomC{$\Gamma \vdash b : \bot$}
\UnaryInfC{$\Gamma \vdash absurd_\varphi\,b : \varphi$}
\DisplayProof
\qquad
\end{center}

\begin{center}
\AxiomC{$\Gamma, x : \varphi \vdash b : \psi$}
\UnaryInfC{$\Gamma \vdash \lambda x.\,b : \varphi \to \psi$}
\DisplayProof
\qquad
\AxiomC{$\Gamma \vdash f : \varphi \to \psi$}
\AxiomC{$\Gamma \vdash a : \varphi$}
\BinaryInfC{$\Gamma \vdash f\,a : \psi$}
\DisplayProof
\end{center}

\begin{center}
\AxiomC{$\Gamma \vdash a : \varphi$}
\AxiomC{$\Gamma \vdash b : \psi$}
\BinaryInfC{$\Gamma \vdash (a,b) : \varphi \times \psi$}
\DisplayProof
\qquad
\AxiomC{$\Gamma \vdash p : \varphi \times \psi$}
\UnaryInfC{$\Gamma \vdash fst\,p : \varphi$}
\DisplayProof
\qquad
\AxiomC{$\Gamma \vdash p : \varphi \times \psi$}
\UnaryInfC{$\Gamma \vdash snd\,t : \psi$}
\DisplayProof
\end{center}

\begin{center}
\AxiomC{$\Gamma \vdash a : \varphi$}
\UnaryInfC{$\Gamma \vdash Left\,a : \varphi \amalg \psi$}
\DisplayProof
\qquad
\AxiomC{$\Gamma \vdash b : \psi$}
\UnaryInfC{$\Gamma \vdash Right\,b : \varphi \amalg \psi$}
\DisplayProof
\end{center}

\begin{center}
\AxiomC{$\Gamma \vdash e : \varphi \amalg \psi$}
\AxiomC{$\Gamma, x : \varphi \vdash a : \chi$}
\AxiomC{$\Gamma, y : \psi \vdash b : \chi$}
\TrinaryInfC{$\Gamma \vdash \mathbf{case}\,e\,\mathbf{of}\,\{\,Left(x) \to a; Right(y) \to b\,\} : \chi$}
\DisplayProof
\end{center}

А теперь задания:

\begin{enumerate}

\item Между правилами вывода логики и конструкциями в лямбда исчислении существует естественная биекция.
    Например, $\to\!\!I$ соответствует абстракции, а $\to\!\!E$ соответствует аппликации.
    Запишите эту биекцию для остальных правил и конструкций.
    
\textbf{Решение.} 

\begin{align*}
	\land I &\Leftrightarrow (,) \\
	\land E_1 &\Leftrightarrow snd \\
	\land E_2 &\Leftrightarrow fst \\
	\lor I_1 &\Leftrightarrow Left \\
	\lor I_2 &\Leftrightarrow Right \\
	\lor E &\Leftrightarrow case .. of
\end{align*}


\item Приведите для следующих теорем деревья вывода и термы, доказывающие их:

Тут только деревья вывода, все термы в \textit{\textbf{ha5.hs}}
\begin{enumerate}
\item $P \to P$

\textbf{Решение.}
\begin{center}
	\AxiomC{}
	\UnaryInfC{$P \vdash P$}
	\RightLabel{ $\to I$}
	\UnaryInfC{$\vdash P \to P$}
	\DisplayProof
\end{center}

\item $P \to (P \to Q) \to Q$

\textbf{Решение.}
\begin{center}
	\AxiomC{}
	\UnaryInfC{$P , (P \to Q) \vdash P$}
	\RightLabel{ $\to I$}
	\UnaryInfC{$P \vdash (P \to Q) \to P$}
	\RightLabel{ $\to I$}
	\UnaryInfC{$\vdash P \to (P \to Q) \to P$}
	\DisplayProof
\end{center}
\item $P \land Q \to P \lor Q$
\begin{center}
	\AxiomC{}
	\UnaryInfC{$ P \land Q \vdash P \land Q $}
	\RightLabel{ $ \land E_1 $}
	\UnaryInfC{$ P \land Q \vdash P $}
	\RightLabel{ $ \lor I_1 $}
	\UnaryInfC{$ P \land Q \vdash P \lor Q $}
	\RightLabel{ $\to I$}
	\UnaryInfC{$ \vdash P \land Q \to P \lor Q$}
	\DisplayProof
\end{center}
\item $(P \lor Q) \land R \to (P \land R) \lor (Q \land R)$
\begin{center}
\AxiomC{}
\UnaryInfC{$\Gamma \vdash (P \lor Q) \land R $}
\RightLabel{$\land E_1$}
\UnaryInfC{$\Gamma \vdash P \lor Q$}
\AxiomC{}
\UnaryInfC{$\Gamma, P \vdash P $}
\AxiomC{}
\UnaryInfC{$\Gamma \vdash (P \lor Q) \land R $}
\RightLabel{ $\land E_2$}
\UnaryInfC{$\Gamma, Q \vdash R $}
\RightLabel{ $\land I$}
\BinaryInfC{$\Gamma, P \vdash P \land R $}
\RightLabel{ $\lor I_1$}
\UnaryInfC{$\Gamma, P \vdash (P \land R) \lor (Q \land R) $}
\AxiomC{}
\UnaryInfC{$\Gamma, Q \vdash Q $}
\AxiomC{}
\UnaryInfC{$\Gamma \vdash (P \lor Q) \land R $}
\RightLabel{ $\land E_2$}
\UnaryInfC{$\Gamma, Q \vdash R $}
\RightLabel{ $\land I$}
\BinaryInfC{$\Gamma, Q \vdash Q \land R $}
\RightLabel{ $\lor I_2$}
\UnaryInfC{$\Gamma, Q \vdash (P \land R) \lor (Q \land R) $}
\RightLabel{ $\lor E$}
\TrinaryInfC{$\Gamma \vdash (P \land R) \lor (Q \land R)$}
\RightLabel{ $\to I$}
\UnaryInfC{$\vdash (P \lor Q) \land R \to (P \land R) \lor (Q \land R)$}
\DisplayProof
\end{center}

Где $\Gamma = (P \lor Q) \land R$
\end{enumerate}

\item Приведите для следующих теорем доказывающие их термы:

\textbf{Решение}.

Для всех пунктов см. \textit{\textbf{ha5.hs}}
\begin{enumerate}
\item $(P \land R) \lor (Q \land R) \to (P \lor Q) \land R$
\item $(P \lor Q) \lor R \to P \lor (Q \lor R)$
\item $((((P \to Q) \to P) \to P) \to Q) \to Q$
\end{enumerate}

\item Добавим в нашей логике новую связку $\leftrightarrow$, удовлетворяющую следующим условиям:
\begin{align*}
\top \leftrightarrow \top & = \top \\
\top \leftrightarrow \bot & = \bot \\
\bot \leftrightarrow \top & = \bot \\
\bot \leftrightarrow \bot & = \top
\end{align*}
\begin{enumerate}
\item Опишите правила введения и элиминации для этой связки.
    Они не должны использовать никакие другие связки.
    
    \textbf{Решение.}
\begin{center}
	\AxiomC{$\Gamma \vdash \varphi \leftrightarrow \psi$}
	\AxiomC{$\Gamma \vdash \varphi$}
	\RightLabel{ $\leftrightarrow E_1$}
	\BinaryInfC{$\Gamma \vdash \psi$}
	\DisplayProof
	\qquad
	\AxiomC{$\Gamma \vdash \varphi \leftrightarrow \psi$}
	\AxiomC{$\Gamma \vdash \psi$}
	\RightLabel{ $\leftrightarrow E_1$}
	\BinaryInfC{$\Gamma \vdash \varphi$}
	\DisplayProof
\end{center}

\begin{center}
	\AxiomC{$\Gamma, \varphi \vdash \psi$}
	\AxiomC{$\Gamma, \psi \vdash \varphi$}
	\RightLabel{ $\leftrightarrow I$}
	\BinaryInfC{$\Gamma \vdash \varphi \leftrightarrow \psi$}
	\DisplayProof
\end{center}
    
    
    
    
\item Опишите аналогичные конструкции и правила типизации для них в лямбда исчислении.
\item Приведите терм, доказывающий формулу $(P \lor Q \to R) \leftrightarrow (P \to R) \land (Q \to R)$.
\end{enumerate}

\end{enumerate}



Пусть $M$ -- частично упорядоченное множество и $x,y \in M$.
Тогда \emph{супремум} $x$ и $y$ -- это элемент $x \lor y \in M$, удовлетворяющий следующим условиям:
\begin{itemize}
\item $x \leq x \lor y$.
\item $y \leq x \lor y$.
\item Для любого $\gamma$ если $x \leq \gamma$ и $y \leq \gamma$, то $x \lor y \leq \gamma$.
\end{itemize}
Супремум двух элементов может не существовать, но если он существует, то он уникален.

\emph{Инфимум} $x$ и $y$ -- это элемент $x \land y \in M$, удовлетворяющий следующим условиям:
\begin{itemize}
\item $x \land y \leq x$.
\item $x \land y \leq y$.
\item Для любого $\gamma$ если $\gamma \leq x$ и $\gamma \leq y$, то $\gamma \leq x \land y$.
\end{itemize}
Инфимум двух элементов может не существовать, но если он существует, то он уникален.

Эти условия в определениях супремума и инфимума можно записать в следующем виде (это те же самые условия, но в 
такой форме их может быть проще воспринимать):
\begin{center}
\AxiomC{}
\UnaryInfC{$x \leq x \lor y$}
\DisplayProof
\qquad
\AxiomC{}
\UnaryInfC{$y \leq x \lor y$}
\DisplayProof
\qquad
\AxiomC{$x \leq \gamma$}
\AxiomC{$y \leq \gamma$}
\BinaryInfC{$x \lor y \leq \gamma$}
\DisplayProof
\end{center}

\begin{center}
\AxiomC{}
\UnaryInfC{$x \land y \leq x$}
\DisplayProof
\qquad
\AxiomC{}
\UnaryInfC{$x \land y \leq y$}
\DisplayProof
\qquad
\AxiomC{$\gamma \leq x$}
\AxiomC{$\gamma \leq y$}
\BinaryInfC{$\gamma \leq x \land y$}
\DisplayProof
\end{center}

\begin{enumerate}

\item Закончите доказательство того, что интерпретация логики с $\land$ и $\lor$ в дистрибутивных решетках 
корректна.
    То есть нужно доказать, что если $\gamma \leq \varphi \lor \psi$, $\gamma \land \varphi \leq \chi$ и $\gamma \land \psi \leq \chi$, то $\gamma \leq \chi$.
    
    \textit{Решена в классе}

\item Докажите, что в любой решетке для любых элементов $x$, $\varphi$ и $\psi$ следующие утверждения эквивалентны:
\begin{enumerate}
\item Для любого $\gamma$ если $\gamma \leq x$, то $\gamma \land \varphi \leq \psi$.
\item Для любого $\gamma$ если $\gamma \leq x$ и $\gamma \leq \varphi$, то $\gamma \leq \psi$.
\item $x \land \varphi \leq \psi$.
\end{enumerate}

\textit{Решена в классе}

\item Пусть $\varphi$, $\psi$, $x$ и $x'$ -- элементы решетки. Допустим в ней верны следующие свойства:
\begin{enumerate}
\item Для любого $\gamma$
    \[ \gamma \leq x \Leftrightarrow \gamma \land \varphi \leq \psi \]
\item Для любого $\gamma$
    \[ \gamma \leq x' \Leftrightarrow \gamma \land \varphi \leq \psi \]
\end{enumerate}
Докажите, что тогда $x = x'$.

\textbf{Решение.} 

Подставим в $(a)$ вместо $\gamma$ элемент $x$. Очевидно, $x \leq x$. Значит, верно, что $x \land \varphi \leq 
\psi$. Заметим, что это совпадает с левой частью $(b)$. Значит, $x \leq x'$.

Можно провести аналогичные рассуждения для $x'$ и $(b)$, получив $x' \leq x$. 
\begin{equation*}
	(x \leq x') \land (x' \leq x) \Rightarrow (x = x')
\end{equation*}

\item Покажите, что в любой алгебре Гейтинга $M$ верны следующие свойства:
\begin{enumerate}
\item В $M$ существует наибольший элемент, то есть элемент $\top$, удовлетворяющий условию, что $x \leq \top$ для 
любого $x$.
\item Для любых $\varphi, \psi \in M$ верно $\varphi \leq \psi \Leftrightarrow (\varphi \to \psi) = \top$
\end{enumerate}

\textit{Решена в классе}

\item Докажите, что любая алгебра Гейтинга дистрибутивна.
    Hint: Нужно отдельно доказать, что
    $(\varphi \land \psi) \lor (\varphi \land \chi) \leq \varphi \land (\psi \lor \chi)$ и
    $\varphi \land (\psi \lor \chi) \leq (\varphi \land \psi) \lor (\varphi \land \chi)$.
    Первое свойство верно в любой решетке, при доказательстве второго используется, что решетка ялвяется алгеброй Гейтинга.
    
    \textbf{Решение.}
    
    Сначала покажем, что $(\varphi \land \psi) \lor (\varphi \land \chi) \leq \varphi \land (\psi \lor \chi)$.
    \begin{center}
    	\AxiomC{}
    	\UnaryInfC{$\varphi \land \psi \leq \varphi$}
    	\AxiomC{}
    	\UnaryInfC{$\varphi \land \psi \leq \psi \leq \psi \lor \chi$}
    	\UnaryInfC{$\varphi \land \psi \leq \psi \lor \chi$}
    	\BinaryInfC{$\varphi \land \psi \leq \varphi \land (\psi \lor \chi)$}
    	\AxiomC{}
    	\UnaryInfC{$\varphi \land \chi \leq \varphi$}
    	\AxiomC{}
    	\UnaryInfC{$\varphi \land \chi \leq \chi \leq \psi \lor \chi$}
    	\UnaryInfC{$\varphi \land \chi \leq \psi \lor \chi$}
    	\BinaryInfC{$\varphi \land \chi \leq \varphi \land (\psi \lor \chi)$}
    	\BinaryInfC{$(\varphi \land \psi) \lor (\varphi \land \chi) \leq \varphi \land (\psi \lor \chi)$}
    	\DisplayProof
    \end{center}
    
    И теперь \textit{обратно}, $\varphi \land (\psi \lor \chi) \leq (\varphi \land \psi) \lor (\varphi \land 
    \chi)$.
    
    Прежде всего:
    \begin{align*}
    	\psi \land \varphi &\leq (\varphi \land \psi) \lor (\varphi \land \chi) \Leftrightarrow \psi \leq (\varphi \to (\varphi \land \psi) \lor (\varphi \land \chi)) \\
    	\chi \land \varphi &\leq (\varphi \land \psi) \lor (\varphi \land \chi) \Leftrightarrow \chi \leq (\varphi \to (\varphi \land \psi) \lor (\varphi \land \chi))
    \end{align*}
    Воспользуемся $(\lor E)$ и свойством алгебры Гейтинга:
    \begin{equation*}
    	\chi \lor \psi \leq (\varphi \to (\varphi \land \psi) \lor (\varphi \land \chi)) \Leftrightarrow
    	(\chi \lor \psi) \land \varphi \leq (\varphi \land \psi) \lor (\varphi \land \chi)
    \end{equation*}
    
    Это то, что мы и хотели.

\item Докажите, что в любом частично упорядоченном множестве следующие условия эквивалентны:
\begin{enumerate}
\item Для любого множества $S$ ее элементов существует их супремум $\bigvee S$.
\item Для любого множества $S$ ее элементов существует их инфимум $\bigwedge S$.
\end{enumerate}

	\textbf{Решение.}
	
	%$(a) \Rightarrow (b)$. Существует супремум $\Rightarrow$ соответствующая решётка полна $\Rightarrow$ в ней %существует наименьший элемент $\bot = \bigwedge \varnothing$
	
	%$(b) \Rightarrow (a)$. 

\item Докажите, что следующие формулы не выводимы в пропозициональной логике:
\begin{enumerate}
\item $(P \to Q) \to \neg P \lor Q$.

	\textbf{Решение.}
	
	Выберем $P = (0, 2), Q = (1, 3)$. Теперь 
	\begin{align*}
		P \to Q &= (-\infty; 0) \cup (1; +\infty) \\
		\neg P \lor Q &= (-\infty; 0) \cup (1; +\infty) \\
		(P \to Q) \to (\neg P \lor Q) &= (-\infty; 0) \cup (0, 1) \cup (1, +\infty) \neq \mathbb{R}
	\end{align*}
\item $\neg (P \land Q) \to \neg P \lor \neg Q$.
	\textbf{Решение.}
	
	Выберем $P = (0, 3), Q = (1, 2)$. Теперь
	\begin{align*}
		\neg (P \land Q) &= (-\infty; 1) \cup (2, +\infty) \\
		\neg P \lor \neg Q &= (-\infty; 1) \cup (2, +\infty) \\
		\neg (P \land Q) \to \neg P \lor \neg Q &= (-\infty; 1) \cup (1, 2) \cup (2, +\infty) \neq \mathbb{R}
	\end{align*}

\end{enumerate}

\end{enumerate}


\section*{Классическая логика}

\begin{enumerate}

\item Докажите, что формула $((P \to Q) \to P) \to P$ невыводима, показав, что ее выводимость влечет выводимость 
исключенного третьего.

\textbf{Решение.}
Предположим, что это правда, тогда правдой является и формула (по лемме об уточнении): $(((S\lor \neg S) \to 
\bot) \to (S \lor \neg S)) \to (S \lor \neg S)$. (Формула полученной подстановкой $P = (S \lor \neg S), Q = 
\bot$). 

Заметим, что в этом предположении выводится исключённое третье (обозначим $\varTheta = (((S\lor \neg S) \to \bot) 
\to (S \lor \neg S)) \to (S \lor \neg S)$):
\begin{center}
	\AxiomC{}
	\UnaryInfC{$\vdash \varTheta$}
	\AxiomC{}
	\UnaryInfC{$ ((S\lor \neg S) \to \bot), S \vdash ((S\lor \neg S) \to \bot) $}
	\AxiomC{}
	\UnaryInfC{$ ((S\lor \neg S) \to \bot), S \vdash S $}
	\UnaryInfC{$ ((S\lor \neg S) \to \bot), S \vdash (S \lor \neg S) $}
	\BinaryInfC{$ ((S\lor \neg S) \to \bot), S \vdash \bot $}
	\UnaryInfC{$ ((S\lor \neg S) \to \bot) \vdash S \to \bot $}
	\UnaryInfC{$ ((S\lor \neg S) \to \bot) \vdash \neg S $}
	\UnaryInfC{$ ((S\lor \neg S) \to \bot) \vdash (S \lor \neg S) $}
	\UnaryInfC{$\vdash ((S\lor \neg S) \to \bot) \to (S \lor \neg S) $}
	\BinaryInfC{$\vdash S \lor \neg S$}
	\DisplayProof
\end{center}

Этого не может быть, значит, мы получили противоречие.

\item Докажите, что в интуиционистской логике следующие правила вывода эквивалентны:
\begin{center}
\AxiomC{}
\RightLabel{, (lem)}
\UnaryInfC{$\Gamma \vdash \varphi \lor \neg \varphi$}
\DisplayProof
\qquad
\AxiomC{}
\RightLabel{, (dne)}
\UnaryInfC{$\Gamma \vdash \neg \neg \varphi \to \varphi$}
\DisplayProof
\end{center}
Для этого достаточно показать, что имея правило (lem), заключение правила (dne) выводимо, и наоборот.

\textbf{Решение.} $lem \Rightarrow dne$. Честно построим дерево вывода.
	\begin{center}
		\AxiomC{}
		\UnaryInfC{$\vdash \varphi \lor \neg \varphi$}
		\AxiomC{}
		\UnaryInfC{$\neg \varphi ,(\neg\varphi \to \bot) \vdash \neg\varphi \to \bot$}
		\AxiomC{}
		\UnaryInfC{$\neg \varphi ,(\neg\varphi \to \bot) \vdash \neg\varphi$}
		\BinaryInfC{$\neg \varphi ,(\neg\varphi \to \bot) \vdash \bot$}
		\UnaryInfC{$\neg \varphi ,(\neg\varphi \to \bot) \vdash \varphi$}
		\UnaryInfC{$\neg \varphi \vdash (\neg\varphi \to \bot) \to \varphi$}
		\AxiomC{}
		\UnaryInfC{$\varphi, (\neg\varphi \to \bot) \vdash \varphi$}
		\UnaryInfC{$\varphi \vdash (\neg\varphi \to \bot) \to \varphi$}
		\TrinaryInfC{$\vdash (\neg\varphi \to \bot) \to \varphi$}
		\UnaryInfC{$\vdash \neg \neg \varphi \to \varphi$}
		\DisplayProof
	\end{center}
	$dne \Rightarrow lem$	
	\begin{center}
		\AxiomC{}
		\UnaryInfC{$\vdash \neg \neg (\varphi \lor \neg \varphi) \to \varphi \lor \neg \varphi$}
		\AxiomC{$\vdash \neg \neg (\varphi \lor \neg \varphi)$}
		\BinaryInfC{$\vdash \varphi \lor \neg \varphi$}
		\DisplayProof
	\end{center}
	
	Заметим, что выражение слева следует из $dne$. Осталось доказать лишь выражение справа. Сделаем это с помощью 
	терма, которому можно приписать тип $(\varphi \lor (\varphi \to \bot)) \to \bot) \to \bot$:
	\begin{equation*}
		\lambda f. f \ (Right \ (\lambda x. f (Left \ x))) 
	\end{equation*}

	

\item Формула $\varphi \land \psi$ доказуема тогда и только тогда когда доказуемы $\varphi$ и $\psi$. Верно ли 
аналогичное утверждение для $\varphi \lor \psi$?  То есть, правда ли что $\varphi \lor \psi$ доказуема тогда и 
только тогда когда доказуема либо $\varphi$, либо $\psi$?В классической логике это не верно. Действительно, $P 
\lor \neg P$ доказуема, но ни $P$, ни $\neg P$ не доказуемы. Докажите, что в интуиционистской логике это свойство 
выполнено.
    
    Hint: используйте в качестве доказательств формул лямбда термы и примените нормализацию.

\item Докажите в классической логике формулу $\neg (\varphi \land \psi) \to \neg \varphi \lor \neg \psi$
    Напишите лямбда терм, доказывающий эту формулу.
    
    \textbf{Решение.}
    В классический логике нам достаточно перебрать значения переменных $\varphi, \psi$, чтобы доказать формулу.
   
   \begin{longtable}{c|c|c|c|c}
   		$\varphi$ & $\psi$ & $\neg(\varphi \land \psi)$ & $\neg \varphi \lor \neg \psi$ & $\neg(\varphi \land 
   		\psi) \to \neg \varphi \lor \neg \psi$  \\ \hline
   		0 & 0 & 1 & 1 & 1 \\ 
   		0 & 1 & 1 & 1 & 1 \\ 
   		1 & 0 & 1 & 1 & 1 \\ 
   		1 & 1 & 0 & 0 & 1 \\ 
   	\end{longtable}
   	
   	Терм, который это доказывает:
   	\begin{align*}
	   	term \ f = case \ lem_{\varphi} \ of \ \{Right \ f' \to Left \ f'; Left \ \varphi' \to Right \ f \ . \ 
	   	(\psi \to (\varphi', \psi))\}
   	\end{align*}

\item Докажите в интуиционистской логике формулу $(((P \to Q) \lor (Q \to P)) \to P) \to P$.
    Напишите лямбда терм, доказывающий эту формулу.

	\textbf{Решение.}
	
	Формула доказывается, например, таким термом (терм \textit{task5} в файле \textbf{proofs.hs}):
	\begin{equation*}
		\lambda f. f(Right \ (\lambda q. f (Left \ (\lambda p.q))))
	\end{equation*}

\item Реализуйте алгоритм, возвращающей по формуле ее доказательство в классической логике.

\item Реализуйте алгоритм, возвращающей по классическому доказательству формулы $\neg \psi$ ее интуиционистское 
доказательство.

\end{enumerate}

\section*{Логика первого порядка}

\begin{enumerate}

\item Определите формулы, удовлетворяющие следующим описаниям.
    Для первых двух заданий мы предполагаем, что в сигнатуре есть предикатный символ $\leq$.
\begin{enumerate}
\item $\exists x \leq t\ \psi$ (``Существует $x$, меньше либо равный $t$, такой что верно $\psi$'').
\item $\forall x \leq t\ \psi$ (``Для любого $x$, меньше либо равного $t$, верно $\psi$'').
\item ``Существует не менее двух элементов, удовлетворяющих $\varphi(x)$''.
\begin{equation*}
\exists x \exists y \ (x \neq y) \land \varphi(x) \land \varphi(y)
\end{equation*}
\item ``Существует ровно два элемента, удовлетворяющие $\varphi(x)$''.
\begin{equation*}
	\exists x \exists y \ (x \neq y) \land \varphi(x) \land \varphi(y) \land (\forall z \ \varphi(z) \to (z = x \lor z = y))
\end{equation*}
\item ``Существует по крайней мере один, но не более двух элементов, удовлетворяющих $\varphi(x)$''.
\begin{equation*}
	\exists x \ \varphi(x) \land (\forall p \forall y \forall z \ \varphi(p) \land \varphi(y) \land \varphi(z) \to ((p = y) \lor (y = z) \lor (p = z)))
\end{equation*}

\item ``Существует не более одного элемента, удовлетворяющего $\varphi(x)$''
\begin{equation*}
	\forall x \forall y \ (\varphi(x) \land \varphi(y) \to (x = y)
\end{equation*}
\end{enumerate}

\item Напишите на хаскелле функцию, аналогичную конструкции $\mathbf{case}$ для пар, используя $fst$ и $snd$.
    Укажите ее тип (вам нужно будет использовать функции высшего порядка вместо расширения контекста).
    Реализуйте функции $fst'$ и $snd'$, эквивалентные обычным $fst$ и $snd$, через эту функцию.

\item Пусть у нас есть несколько формул:
\begin{enumerate}
\item \label{it:no} $x \neq y$ -- $\{M_0\}$
\item \label{it:e} $\exists x (x \neq y)$ -- $\{\}$
\item \label{it:a} $\forall x (x \neq y)$ -- $\{M_0\}$
\item \label{it:ee} $\exists x \exists y (x \neq y)$ -- $\{M_2\}$
\item \label{it:ea} $\exists x \forall y (x \neq y)$ -- $\{\}$
\item \label{it:ae} $\forall x \exists y (x \neq y)$ -- $\{M_2\}$
\item \label{it:aa} $\forall x \forall y (x \neq y)$ -- $\{M_0\}$
\end{enumerate}
И несколько интерпретаций:
\begin{align*}
M_0 & = \varnothing \\
M_1 & = \{ 7 \} \\
M_2 & = \{ 13, 28 \}
\end{align*}
Какие из этих формул верны в каких моделях?

\item Докажите, что формулы $\forall x \forall y (x \neq y)$ и $\neg \exists x\ \top$ эквивалентны,
    написав лямбда терм типа $((\forall x \forall y (x \neq y)) \to \neg \exists x\ \top) \land ((\neg \exists x\ \top) \to \forall x \forall y (x \neq y))$.

\item Пусть теория $T$ состоит из одной аксиомы $\{\,st : \exists (x : s)\ \top\,\}$.
    Пусть $x \notin FV(\varphi)$.
    Тогда докажите, что следующие формулы являются теоремами этой теории.
    Приведите и дерево вывода, и лямбда терм, доказывающие эти формулы.
\begin{enumerate}
\item $\varphi \land (\forall x\ \psi) \to \forall x\ (\varphi \land \psi)$ (эта формула выводима и в пустой теории)
\item $(\forall x\ (\varphi \land \psi)) \to \varphi \land (\forall x\ \psi)$
\end{enumerate}

\item Докажите, что следующие формулы выводимы (в пустой теории), написав лямбда термы, доказывающие их.
    Исключенное третье можно (и нужно) использовать только в последнем пункте.
\begin{enumerate}
\item $(\forall x\ (\neg \varphi)) \to \neg \exists x\ \varphi$
\item $(\neg \exists x\ \varphi) \to \forall x\ (\neg \varphi)$
\item $\exists x\ (\neg \varphi) \to \neg \forall x\ \varphi$
\item $(\neg \forall x\ \varphi) \to \exists x\ (\neg \varphi)$
\end{enumerate}

\end{enumerate}


\section*{Формальная арифметика}

В заданиях, где требуется привести доказательство, нужно привести словесное описание доказательства.
Необходимо явно указать, где и какие аксиомы арифметики Пеано были использованы.
Все доказательства должны быть в интуиционистской логике.
\begin{enumerate}
	
\item Определите формулу $\varphi(x,y)$, задающую график функции $pred$, удовлетворяющей следующим условиям:
\begin{align*}
pred(0) & = 0 \\
pred(S(x)) & = x
\end{align*}
Докажите, что $\forall x \exists! y (\varphi(x,y))$

\textbf{Решение.}
\begin{align*}
	\varphi(0, y) &= 0 = y \\
	\varphi(S(x), y) &= x = y
\end{align*}

Утверждение $\forall x \exists! y (\varphi(x,y))$ следует из единственности элементов:
\begin{align*}
	\forall &x \ 0 \neq S(x)\\
	\forall &x \forall y \ S(x) = S(y) \to x = y
\end{align*}

\item Докажите, что аксиомы для сложения определяют его уникальным образом.
    То есть если мы добавим в сигнатуру новый функциональный символ $+'$ и новые аксиомы
\begin{align*}
\forall y\ & 0 +' y = y \tag{$+'0$} \\
\forall x \forall y\ & S(x) +' y = S(x +' y) \tag{$+'S$},
\end{align*}
то в ней будет доказуема формула $\forall x \forall y\ (x + y = x +' y)$.

\textbf{Решение.} 

Чтобы доказать эту формулу (обозначим её $\varphi(x) = \forall y\ (x + y = x +' y)$). Нам достаточно показать, что $\varphi(0)$ верно, и $\forall x \ \varphi(x) \to \varphi(S(x))$. Покажем, что это действительно так.

$\varphi(0):$ 
\begin{equation*}
	\forall y \ (0 + y = 0 +' y) \to \forall y \ (y = y)
\end{equation*}
Выполнено.

Покажем второй пункт  $\forall x \ \varphi(x) \to \varphi(S(x))$:
\begin{equation*}
	\forall x (\forall y (x + y = x +' y) \to \forall y (S(x) + y = S(x) +' y))
\end{equation*}
Преобразуем $S(x) + y = S(x) +' y$:
\begin{equation*}
	S(x) + y = S(x) +' y \to S(x + y) = S(x +' y) \to x + y = x +' y
\end{equation*}
По индукционному предположению, последнее выражение верно, поэтому всё доказано.

\item Добавим в сигнатуру функциональные символы $exp$ и $fac$ для функций возведения в степень и факториала
соответственно.
    Напишите аксиомы для этих функциональных символов, определяющие их уникальным образом.

	\textbf{Решение.}
	\begin{align*}
		x \ exp \ 0 &= S(0)\\
		x \ exp \ S(y) &= x \cdot (x \ exp\ y) \\ 
		fac \ 0 &= S(0) \\
		fac \ S(x) &= S(x) \cdot fac\ x
	\end{align*}

\item Докажите следующие свойства:
\begin{enumerate}
\item $\forall x \forall y\ (x + y = 0 \to x = 0 \land y = 0)$.

\textbf{Решение.} 

Обозначим $\varphi(x) = \forall y (x + y = 0 \to x = 0 \land y = 0)$. Для доказательства этого утверждения 
достаточно показать, что верно $\varphi(0)$ и $\forall x (\varphi(x) \to \varphi(S(x))$.

$\varphi(0):$ 
\begin{equation*}
\forall y \ ((0 + y = 0) \to (0 = 0 \land y = 0)) \to \forall (y = 0 \to y = 0)
\end{equation*}
Выполнено.

Покажем второй пункт  $\forall x \ \varphi(x) \to \varphi(S(x))$:
\begin{equation*}
\forall x (\forall y [x + y = 0 \to x = 0 \land y = 0] \to \forall y [S(x) + y = 0 \to  = S(x) = 0 \land y = 0])
\end{equation*}

Рассмотрим выражение $S(x) + y = 0 \to  = S(x) = 0 \land y = 0$, Заметим, что оно преобразуется к виду $S(x + 
y) = 0 \to S(x) = 0 \land y = 0$, или что то же самое $\bot \to (S(x) = 0 \land y = 0)$. Которое очевидно 
верно, т.к. из $\bot$ может следовать любое утверждение. 

Значит верным является и $\forall x \ \varphi(x) \to \varphi(S(x))$, что вместе с $\varphi(0)$ доказывает 
исходное утверждение.

\item $\forall x \forall y\ (x \cdot y = 0 \to x = 0 \lor y = 0)$.

\textbf{Решение.} 

Обозначим $\varphi(x) = \forall y x \cdot y = 0 \to x = 0 \lor y = 0)$. Для доказательства этого утверждения 
достаточно показать, что верно $\varphi(0)$ и $\forall x (\varphi(x) \to \varphi(S(x))$.

$\varphi(0):$ 
\begin{equation*}
(\forall y \ 0 \cdot y = 0 \to 0 = 0 \lor y = 0) \to (\forall y [y = 0 \to \top]) \to \top 
\end{equation*}
Выполнено.

Покажем второй пункт  $\forall x \ \varphi(x) \to \varphi(S(x))$:
\begin{equation*}
\forall x (\forall y [x \cdot y = 0 \to x = 0 \lor y = 0] \to \forall y [S(x) \cdot y = 0 \to S(x) = 0 \lor y = 0])
\end{equation*}

Рассмотрим выражение $S(x) \cdot y = 0 \to S(x) = 0 \lor y = 0$, Заметим, что оно преобразуется к виду $y + xy = 0 \to S(x) = 0 \lor y = 0$. Воспользуемся первым пунктом, тогда вместо этого выражения достаточно доказать: $y = 0 \land xy = 0 \to S(x) = 0 \lor y = 0$. Заметим, что это действительно так: 
\begin{equation*}
(y = 0 \land xy = 0) \to (y = 0) \to (S(x) = 0 \lor y = 0)
\end{equation*}

Значит верным является и $\forall x \ \varphi(x) \to \varphi(S(x))$, что вместе с $\varphi(0)$ доказывает 
исходное утверждение.
\end{enumerate}

\item Докажите коммутативность сложения.

\textbf{Решение.} Нужно доказать, что.
\begin{equation*}
\forall x \forall y \ x + y = y + x
\end{equation*}

Выберем $\varphi(x) = \forall y \ x + y = y + x$. Докажем $\varphi(x)$ по индукции.

$\varphi(0) = \forall y \ 0 + y = y + 0$.

Это утверждения тоже докажем по индукции $\psi(y) = 0 + y = y + 0$.
База индукции очевидна: $0 + 0 = 0 = 0 + 0$.

Докажем индукционный переход. $\forall y \ \psi(y) \to \psi(S(y))$.

Пусть мы знаем, что верно $0 + y = y + 0$. Покажем, что $0 + S(y) = S(y) + 0$.
\begin{equation*}
	0 + S(y) = S(y) = S(0 + y) = S(y + 0) = S(y) + 0
\end{equation*}
Доказано.

Таким образом, доказали базу индукции для коммутативности сложения. Остался индукционный переход $\forall x \ \varphi(x) \to \varphi(S(x))$

Т.е. если $x + y = y + x$, то $S(x) + y = y + S(x)$.

Выражение $\gamma(y) = S(x) + y = y + S(x)$ тоже докажем по индукции.

База $\gamma(0) = S(x) + 0 = 0 + S(x)$. Это то же самое, что и база самой первой индукции, которую мы уже доказали.

Переход $\gamma(y) \to \gamma(S(y))$. То есть нужно показать, что верно $S(x) + S(y) = S(y) + S(x)$ в предположении $S(x) + y = y + S(x)$. Покажем, что это действительно так.
% TODO: тут что-то стронное написано. Нужно перепроверить.
\begin{align*}
	&S(x) + S(y) = S(x + 0) + S(y) = x + S(0) + S(y) = x + S(0 + S(y)) =\\ &x + S(S(y) + 0) = x + S(y) + S(0) = S(y) + x + S(0) = S(y) + S(x + 0) =\\ &S(y) + S(x)
\end{align*}
Заметим, что только что доказанное утверждение $S(x) + y = y + S(x)$ доказывает индукционный переход для коммутативности сложения. 

\end{enumerate}


\section*{Расшикрения арифметики}

\begin{enumerate}

\item Докажите, что $L_1^{lem}$ является консервативным расширением $L_0^{lem}$.

\textbf{Решение.} 

Достаточно привести эквивалентную формулу в классической логике для выражения с конъюнкцией $\varphi \land \psi$. Можно убедиться (перебрав значения все значения $\varphi, \psi$ в булевой интерпретации $\{\bot,\top\}$), что конъюнкции соответствует формула:
\begin{equation*}
	\neg(\varphi \to \neg \psi)
\end{equation*}

\item Если в $PRCPA$ можно определять функции при помощи паттерн матчинга на нескольких аргументах сразу, то мы легко можем определить функции $min$ и $max$:
\begin{align*}
& min : \mathbb{N} \times \mathbb{N} \to \mathbb{N} \\
& min(0,y) = 0 \\
& min(S(x),0) = 0 \\
& min(S(x),S(y)) = S(min(x,y)) \\
& \\
& max : \mathbb{N} \times \mathbb{N} \to \mathbb{N} \\
& max(0,y) = y \\
& max(S(x),S(y)) = S(max(x,y))
\end{align*}
\begin{enumerate}
\item Определите их, используя только базовый вариант паттерн матчинга как в лекциях.

Сначала введём дополнительную конструкцию:
\begin{align*}
& sub : \mathbb{N} \times \mathbb{N} \to \mathbb{N} \\
& sub (x, 0) = x \\
& sub (x, S(y)) = pred (sub(x, y))
\end{align*}

Очевидно, эта функция является примитивно-рекурсивной. После этого становится возможным ввести и функции $min, max$:

\begin{align*}
& min : \mathbb{N} \times \mathbb{N} \to \mathbb{N} \\
& min(0,y) = 0 \\
& min(S(x), y) = sub(S(x + y), max(S(x), y)) \\
& \\
& max : \mathbb{N} \times \mathbb{N} \to \mathbb{N} \\
& max(0,y) = y \\
& max(S(x), y) = y + sub(S(x), y)
\end{align*}
\item Докажите, что $\forall x \forall y\ (min(x,y) \leq max(x,y))$, где $a \leq b$ означает $\exists c\ (a + c = b)$.
\end{enumerate}

\item Пусть в $PRCPA$ у нас есть функции $+$ и $+'$, определённые следующим образом:
\begin{align*}
& + : \mathbb{N} \times \mathbb{N} \to \mathbb{N} \\
& 0 + y = y \\
& S(x) + y = S(x + y) \\
& \\
& +' : \mathbb{N} \times \mathbb{N} \to \mathbb{N} \\
& x +' 0 = x \\
& x +' S(y) = S(x +' y)
\end{align*}
Докажите, что $\forall x \forall y\ (x + y = x +' y)$

\textbf{Решение.} Заметим, что для $+$ мы уже доказали коммутативность. Если добавить коммутативность в правила 
вычисления термов, то требуемое утверждение доказывается по рефлексивности.

\item Докажите в $CPA$, что $\forall n\ (2^n \leq ack\,n\,n)$, где $a \leq b$ означает $\exists c\ (a + c = b)$ 
и функции $2^{(-)}$ и $ack$ определены следующим образом:
\begin{align*}
& 2^{(-)} : \mathbb{N} \to \mathbb{N} \\
& 2^0 = S(0) \\
& 2^{S(n)} = 2 \cdot 2^n \\
& \\
& ack : \mathbb{N} \to \mathbb{N} \to \mathbb{N} \\
& ack\,0\,n = S(n) \\
& ack\,(S\,m)\,0 = ack\,m\,(S\,0) \\
& ack\,(S\,m)\,(S\,n) = ack\,m\,(ack\,(S\,m)\,n)
\end{align*}

\textbf{Решение.} Заметим, что функция Аккермана монотонна хотя бы по первому аргументу.

Так же, раз мы можем вычислять значения, проверим неравенство для $n = 0, 1, 2$. Оно выполнено:
\begin{itemize}
	\item $2^0 = 1 \le 1 = ack\ 0 \ 0$
	\item $2^1 = 2 \le 3 = ack\ 0 \ 0$
	\item $2^2 = 4 \le 7 = ack\ 0 \ 0$
\end{itemize}

Для остальных $n$ значение $ack \ n \ n$ можно заменить на $ack \ 3 \ n$, которое равно $2^{n + 3} - 3$. 
Очевидно, что $2^n \le 2^{n + 3} - 3$. Так же, в силу монотонности функции аккермана $\forall n \ge 3 \left[ 
ack \ 3 \ n \le ack \ n \ n \right] $. Значит и требуемое неравенство тоже выполнено.

\item Пусть в $CPA$ определена следующая функция:
\begin{align*}
& f : \mathbb{N} \to \mathbb{N} \to \mathbb{N} \\
& f\,0\,x = 2 \cdot x \\
& f\,(S\,n)\,x = f\,n\,(f\,n\,(S\,x))
\end{align*}
Является ли эта функция примитивно рекурсивной? Если да, запишите ее определение (в некарированном виде) в 
$PRCPA$.

\textbf{Решение}
\begin{align*}
& f : \mathbb{N} \times \mathbb{N} \to \mathbb{N} \\
& f \ (0,x) = 2 \cdot x \\
& f \ (n, S(x)) = 2 ^ {2 ^ n} + f \ (n, x) \\
& f \ (S(n), 0) = f \ (n,f \ (n, S(0)))
\end{align*}

\end{enumerate}

\section*{Теория множеств}

\begin{enumerate}

\item Докажите, что отношение $\subseteq$ является частичным порядком. Какие аксиомы при этом необходимо 
использовать?

\textbf{Решение.} Покажем рефлексивность, транзитивность и антисимметричность.
\begin{itemize}
	\item Рефлексивность. $a \subseteq a$. Нужно $\forall z \ (z \in a \to z \in a)$. Это утверждение верное.
	\item Транзитивность. $a \subseteq b \land b \subseteq c \to a \subseteq c$. Покажем, что $\forall z \ z \in a \to z \in c$. 
	\begin{itemize}
		\item $\forall z \ (z \in a \to z \in b)$. Воспользуемся $b \subseteq c$: $(\forall z \ (z \in a \to z \in b 
		\to z \in c)) \to \forall z \ (z \in a \to z \in c)$.
	\end{itemize}
	\item Антисимметричность. $\forall x\forall y (x \subseteq y \land y \subseteq x \to x = y)$. Это аксиома 
	экстенсиональности.
\end{itemize}
\item Мы говорили, что $\in$-индукция влечет тот факт, что не существует бесконечной последовательности множеств 
$\ldots \in x_2 \in x_1 \in x_0$.
    Докажите при помощи $\in$-индукции следующие частные случаи этого утверждения:
\begin{enumerate}
\item Отношение $\in$ иррефлексивно.
\item Не существует $x$ и $y$, таких что $x \in y$ и $y \in x$.
\end{enumerate}

\item Докажите следующие свойства натуральных чисел:
\begin{enumerate}
\item $\forall x \in \mathbb{N} (x = 0 \lor \exists y \in \mathbb{N} (x = S(y)))$.
\item Используя предыдущий пункт и следующий принцип индукции
\[ (\forall x \in \mathbb{N}\ (\forall y < x\ \varphi(y)) \to \varphi(x)) \to \forall x \in \mathbb{N}\ \varphi(x), 
\]
докажите обычный принцип индукции для натуральных чисел:
\[ \varphi(0) \land (\forall y \in \mathbb{N}\ \varphi(y) \to \varphi(S(y))) \to \forall x \in \mathbb{N}\ 
\varphi(x), \]
\end{enumerate}

\item Докажите следующие свойства натуральных чисел (hint: каждый следующий пункт следует из предыдущего):
\begin{enumerate}
\item Если $S(x) < S(y)$, то $x < y$.

\textbf{Решение.}

Нужно показать, что $\left[x \cup \{x\} \in y \cup \{y\}\right] \to x \in y$. Рассмотрим возможные случаи:
\begin{itemize}
	\item $x \cup \{x\} = y$. Из этого сразу следует, что $x \in y$.
	\item $x \cup \{x\} \in y$. Теперь воспользуемся тем, что нам известна структура $y$. Т.е. если $x \cup \{x\} \in 
	y$, то $x \in y$.
\end{itemize}
\item $x < y$ тогда и только тогда, когда $S(x) \leq y$.
\item Отношение $\in$ транзитивно на элементах $\mathbb{N}$.
    Можно показать более общее утверждение: если $x \in y \in z \in \mathbb{N}$, то $x \in z$.
\item Если $x \in y$ и $y \in \mathbb{N}$, то $x \in \mathbb{N}$.
\end{enumerate}

\end{enumerate}


\end{document}
